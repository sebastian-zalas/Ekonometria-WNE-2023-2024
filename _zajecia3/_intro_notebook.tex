% Options for packages loaded elsewhere
\PassOptionsToPackage{unicode}{hyperref}
\PassOptionsToPackage{hyphens}{url}
%
\documentclass[
]{article}
\usepackage{amsmath,amssymb}
\usepackage{lmodern}
\usepackage{iftex}
\ifPDFTeX
  \usepackage[T1]{fontenc}
  \usepackage[utf8]{inputenc}
  \usepackage{textcomp} % provide euro and other symbols
\else % if luatex or xetex
  \usepackage{unicode-math}
  \defaultfontfeatures{Scale=MatchLowercase}
  \defaultfontfeatures[\rmfamily]{Ligatures=TeX,Scale=1}
\fi
% Use upquote if available, for straight quotes in verbatim environments
\IfFileExists{upquote.sty}{\usepackage{upquote}}{}
\IfFileExists{microtype.sty}{% use microtype if available
  \usepackage[]{microtype}
  \UseMicrotypeSet[protrusion]{basicmath} % disable protrusion for tt fonts
}{}
\makeatletter
\@ifundefined{KOMAClassName}{% if non-KOMA class
  \IfFileExists{parskip.sty}{%
    \usepackage{parskip}
  }{% else
    \setlength{\parindent}{0pt}
    \setlength{\parskip}{6pt plus 2pt minus 1pt}}
}{% if KOMA class
  \KOMAoptions{parskip=half}}
\makeatother
\usepackage{xcolor}
\IfFileExists{xurl.sty}{\usepackage{xurl}}{} % add URL line breaks if available
\IfFileExists{bookmark.sty}{\usepackage{bookmark}}{\usepackage{hyperref}}
\hypersetup{
  pdftitle={Rintro},
  pdfauthor={Sebastian Zalas},
  hidelinks,
  pdfcreator={LaTeX via pandoc}}
\urlstyle{same} % disable monospaced font for URLs
\usepackage[margin=1in]{geometry}
\usepackage{color}
\usepackage{fancyvrb}
\newcommand{\VerbBar}{|}
\newcommand{\VERB}{\Verb[commandchars=\\\{\}]}
\DefineVerbatimEnvironment{Highlighting}{Verbatim}{commandchars=\\\{\}}
% Add ',fontsize=\small' for more characters per line
\usepackage{framed}
\definecolor{shadecolor}{RGB}{248,248,248}
\newenvironment{Shaded}{\begin{snugshade}}{\end{snugshade}}
\newcommand{\AlertTok}[1]{\textcolor[rgb]{0.94,0.16,0.16}{#1}}
\newcommand{\AnnotationTok}[1]{\textcolor[rgb]{0.56,0.35,0.01}{\textbf{\textit{#1}}}}
\newcommand{\AttributeTok}[1]{\textcolor[rgb]{0.77,0.63,0.00}{#1}}
\newcommand{\BaseNTok}[1]{\textcolor[rgb]{0.00,0.00,0.81}{#1}}
\newcommand{\BuiltInTok}[1]{#1}
\newcommand{\CharTok}[1]{\textcolor[rgb]{0.31,0.60,0.02}{#1}}
\newcommand{\CommentTok}[1]{\textcolor[rgb]{0.56,0.35,0.01}{\textit{#1}}}
\newcommand{\CommentVarTok}[1]{\textcolor[rgb]{0.56,0.35,0.01}{\textbf{\textit{#1}}}}
\newcommand{\ConstantTok}[1]{\textcolor[rgb]{0.00,0.00,0.00}{#1}}
\newcommand{\ControlFlowTok}[1]{\textcolor[rgb]{0.13,0.29,0.53}{\textbf{#1}}}
\newcommand{\DataTypeTok}[1]{\textcolor[rgb]{0.13,0.29,0.53}{#1}}
\newcommand{\DecValTok}[1]{\textcolor[rgb]{0.00,0.00,0.81}{#1}}
\newcommand{\DocumentationTok}[1]{\textcolor[rgb]{0.56,0.35,0.01}{\textbf{\textit{#1}}}}
\newcommand{\ErrorTok}[1]{\textcolor[rgb]{0.64,0.00,0.00}{\textbf{#1}}}
\newcommand{\ExtensionTok}[1]{#1}
\newcommand{\FloatTok}[1]{\textcolor[rgb]{0.00,0.00,0.81}{#1}}
\newcommand{\FunctionTok}[1]{\textcolor[rgb]{0.00,0.00,0.00}{#1}}
\newcommand{\ImportTok}[1]{#1}
\newcommand{\InformationTok}[1]{\textcolor[rgb]{0.56,0.35,0.01}{\textbf{\textit{#1}}}}
\newcommand{\KeywordTok}[1]{\textcolor[rgb]{0.13,0.29,0.53}{\textbf{#1}}}
\newcommand{\NormalTok}[1]{#1}
\newcommand{\OperatorTok}[1]{\textcolor[rgb]{0.81,0.36,0.00}{\textbf{#1}}}
\newcommand{\OtherTok}[1]{\textcolor[rgb]{0.56,0.35,0.01}{#1}}
\newcommand{\PreprocessorTok}[1]{\textcolor[rgb]{0.56,0.35,0.01}{\textit{#1}}}
\newcommand{\RegionMarkerTok}[1]{#1}
\newcommand{\SpecialCharTok}[1]{\textcolor[rgb]{0.00,0.00,0.00}{#1}}
\newcommand{\SpecialStringTok}[1]{\textcolor[rgb]{0.31,0.60,0.02}{#1}}
\newcommand{\StringTok}[1]{\textcolor[rgb]{0.31,0.60,0.02}{#1}}
\newcommand{\VariableTok}[1]{\textcolor[rgb]{0.00,0.00,0.00}{#1}}
\newcommand{\VerbatimStringTok}[1]{\textcolor[rgb]{0.31,0.60,0.02}{#1}}
\newcommand{\WarningTok}[1]{\textcolor[rgb]{0.56,0.35,0.01}{\textbf{\textit{#1}}}}
\usepackage{longtable,booktabs,array}
\usepackage{calc} % for calculating minipage widths
% Correct order of tables after \paragraph or \subparagraph
\usepackage{etoolbox}
\makeatletter
\patchcmd\longtable{\par}{\if@noskipsec\mbox{}\fi\par}{}{}
\makeatother
% Allow footnotes in longtable head/foot
\IfFileExists{footnotehyper.sty}{\usepackage{footnotehyper}}{\usepackage{footnote}}
\makesavenoteenv{longtable}
\usepackage{graphicx}
\makeatletter
\def\maxwidth{\ifdim\Gin@nat@width>\linewidth\linewidth\else\Gin@nat@width\fi}
\def\maxheight{\ifdim\Gin@nat@height>\textheight\textheight\else\Gin@nat@height\fi}
\makeatother
% Scale images if necessary, so that they will not overflow the page
% margins by default, and it is still possible to overwrite the defaults
% using explicit options in \includegraphics[width, height, ...]{}
\setkeys{Gin}{width=\maxwidth,height=\maxheight,keepaspectratio}
% Set default figure placement to htbp
\makeatletter
\def\fps@figure{htbp}
\makeatother
\setlength{\emergencystretch}{3em} % prevent overfull lines
\providecommand{\tightlist}{%
  \setlength{\itemsep}{0pt}\setlength{\parskip}{0pt}}
\setcounter{secnumdepth}{-\maxdimen} % remove section numbering
\ifLuaTeX
  \usepackage{selnolig}  % disable illegal ligatures
\fi

\title{Rintro}
\author{Sebastian Zalas}
\date{2023-10-18}

\begin{document}
\maketitle

\hypertarget{R-intro}{%
\section{\texorpdfstring{Wprowadzenie do
\texttt{R}}{Wprowadzenie do R}}\label{R-intro}}

setwd('C:/Users/alter-ego/OneDrive/\_ekonometria2023/\_3\_rintro')

\hypertarget{zaczynamy}{%
\subsection{Zaczynamy!}\label{zaczynamy}}

\texttt{R} jest zarówno językiem programowania jak i pakietem
statystycznym, który jest \emph{open-source}. Aby zacząć go używać,
należy zainstalować 2 programy:

\begin{enumerate}
\def\labelenumi{\arabic{enumi}.}
\item
  \href{https://www.r-project.org}{\texttt{R}, właściwy język
  programowania.} - Wybierz właściwy system operacyjny oraz najnowszą
  wersję. - Chose your operating system, and select the most recent
  version.
\item
  \href{http://www.rstudio.com/}{RStudio, IDE - czyli program służący do
  tworzenia oprogramowania w \texttt{R}.} - Zauważ że musisz mieć
  zainstalowany \texttt{R} aby używać RStudio. RStudio to po prostu
  interfejs stosowany do używania \texttt{R}.
\end{enumerate}

\hypertarget{r-jest-szeroko-uzywanym-programem-do-analiz-statystyczych.}{%
\section{\texorpdfstring{\texttt{R} jest szeroko uzywanym programem do
analiz
statystyczych.}{R jest szeroko uzywanym programem do analiz statystyczych.}}\label{r-jest-szeroko-uzywanym-programem-do-analiz-statystyczych.}}

Kilka następnych rozdziałów będzie burzliwym wprowadzeniem do „R''. W
żadnym wypadku nie mają one stanowić kompletnego odniesienia do języka
„R'', ale po prostu wprowadzenie do podstaw, które będą nam potrzebne po
drodze. Kilka ważniejszych tematów zostanie ponownie podkreślonych,
ponieważ są one faktycznie potrzebne do analiz.

Ten wprowadzający rozdział „R'' może wydawać się przytłaczającą ilością
informacji. Nie oczekuje się, że opanujesz wszystko za pierwszym razem.
Powinieneś wypróbować cały kod z tego rozdziału, a następnie wracać do
niego kilka razy, gdy będziesz wracać do koncepcji podczas
przeprowadzania analiz. Przedstawiamy tylko najbardziej podstawowe
aspekty \texttt{R}. Jeśli chcesz dowiedzieć się więcej, istnieje
niezliczona ilość samouczków online i możesz zacząć od oficjalnego
\href{https://cran.r-project.org/doc/manuals/r-release/R-intro.html\#A-sample-session}{CRAN
sample session}lub przejrzyj zasoby na stronie
\href{https://www.rstudio.com/online-learning/\#DataScience}{Rstudio}
lub na tej \href{https://github.com/qinwf/awesome-R}{github repo}.

\hypertarget{starting-r-and-rstudio}{%
\subsection{Starting R and RStudio}\label{starting-r-and-rstudio}}

Kluczową różnicą, którą powinieneś zrozumieć, jest różnica pomiędzy
\texttt{R}, rzeczywistym językiem programowania, a \texttt{RStudio},
popularnym interfejsem \texttt{R}, który pozwala wydajnie i łatwiej
pracować z \texttt{R}.

Najlepszym sposobem na docenienie wartości \texttt{RStudio} jest
rozpoczęcie używania \texttt{R} \emph{bez} \texttt{RStudio}. Aby to
zrobić, kliknij dwukrotnie interfejs GUI R, który powinieneś pobrać na
swój komputer, wykonując powyższe kroki (w systemie Windows lub Mac) lub
uruchom R w terminalu (w systemie Linux lub Mac), wpisując po prostu
„R'' w polu terminala.Właśnie otworzyłeś konsolę R \textbf{konsolę},
która pozwala rozpocząć wpisywanie kodu zaraz po znaku
\texttt{\textgreater{}}, zwaną \emph{podpowiedź}. Spróbuj wpisać
\texttt{2\ +\ 2} lub \texttt{print("Twoje\ imię")} i naciśnij klawisz
Return. I \emph{voilà}, Twoje pierwsze polecenia R!

Wpisywanie w konsoli jednego polecenia po drugim nie jest zbyt wygodne,
ponieważ nasza analiza staje się bardziej skomplikowana. Idealnie
byłoby, gdybyśmy zebrali wszystkie instrukcje poleceń w pliku i
uruchamiali je jedna po drugiej, automatycznie. Możemy to zrobić pisząc
tzw. \textbf{pliki skryptów} lub po prostu \textbf{skrypty}, czyli
proste pliki tekstowe z rozszerzeniem \texttt{.R} lub \texttt{.r}, które
można \emph{wstawić} (lub \emph{pobrać}) w sesję „R''. RStudio bardzo
ułatwia ten proces.

Otwórz \texttt{RStudio}, klikając aplikację \texttt{RStudio} na swoim
komputerze i zwróć uwagę, jak całe środowisko różni się od podstawowej
konsoli \texttt{R} - tak naprawdę ta \emph{taka sama} konsola \texttt{R}
działa w lewym dolnym rogu płyta. Lewy górny panel to miejsce, w którym
możesz pisać skrypty -- czyli wiele linii kodu, które możesz uruchamiać,
kiedy tylko chcesz. Aby uruchomić pojedynczą linię kodu, po prostu
zaznacz ją i naciśnij \texttt{Command} + \texttt{Return}.

RStudio ma dużą liczbę przydatnych skrótów klawiaturowych. Listę z nich
można znaleźć za pomocą skrótu klawiaturowego --- skrótu klawiaturowego,
który pozwala rządzić nimi wszystkimi:

\begin{itemize}
\tightlist
\item
  W systemie Windows: \texttt{Alt} + \texttt{Shift} + \texttt{K}
\item
  Na Macu: \texttt{Opcja} + \texttt{Shift} + \texttt{K}
\end{itemize}

Zespół \texttt{RStudio} opracował {[}wiele „ściągawek''{]}
(\url{https://www.rstudio.com/resources/cheatsheets/}) do pracy zarówno
z \texttt{R}, jak i \texttt{RStudio}.
\href{http://www.rstudio.com/wp-content/uploads/2016/05/base-r.pdf}{Ten
konkretny kod do Base \texttt{R}} podsumowuje wiele koncepcji zawartych
w tym dokumencie. \footnote{Podczas programowania często dobrą praktyką
  jest przestrzeganie przewodnika po stylu. (Gdzie trafiają spacje?
  Tabulatory czy spacje? Podkreślenia lub CamelCase podczas nazywania
  zmiennych?) Żaden przewodnik po stylu nie jest „poprawny'', ale pomaga
  być świadomym tego, co robią inni. Ważniejszą rzeczą jest zachowanie
  spójności w obrębie własnego kodu. Oto dwa przewodniki:
  \href{http://adv-r.had.co.nz/Style.html}{Przewodnik po stylu Hadley
  Wickham} i
  \href{https://google.github.io/styleguide\%20/Rguide.xml}{Przewodnik
  po stylu Google}. W tym kursie naszym głównym odstępstwem od tych
  dwóch wskazówek jest użycie \texttt{=} zamiast \texttt{\textless{}-}.
  Ze względów praktycznych powinieneś pomyśleć \texttt{=} za każdym
  razem, gdy zobaczysz \texttt{\textless{}-}.}

\hypertarget{podstawowe-pojux119cia}{%
\subsubsection{Podstawowe pojęcia}\label{podstawowe-pojux119cia}}

\begin{itemize}
\tightlist
\item
  \texttt{R}: pakiet statystyczny
\item
  \texttt{RStudio}: \emph{integrated development environment (IDE)},
  interfejs do pracy z \texttt{R}
\item
  \emph{komenda}: polecenia (tekst lub liczby) które \texttt{R}
  \emph{rozumie}.
\item
  \emph{skrypt}: lista poleceń zebranych w pliku tekstowym, każda
  oddzielona nową linią, do uruchomienia jedna po drugiej.
\end{itemize}

\hypertarget{podstawowe-obliczenia}{%
\subsection{Podstawowe obliczenia}\label{podstawowe-obliczenia}}

Na początek użyjemy „R'' jak prostego kalkulatora. Uruchom następujący
kod bezpośrednio z konsoli RStudio lub w RStudio, zapisując go w
skrypcie i uruchamiając za pomocą „Polecenie'' + „Return''.

\hypertarget{dodawanie-odejmowanie-mnoux17cenie-i-dzielenie}{%
\paragraph*{Dodawanie, odejmowanie, mnożenie i
dzielenie}\label{dodawanie-odejmowanie-mnoux17cenie-i-dzielenie}}
\addcontentsline{toc}{paragraph}{Dodawanie, odejmowanie, mnożenie i
dzielenie}

\begin{longtable}[]{@{}ccc@{}}
\toprule
Math & \texttt{R} code & Result \\
\midrule
\endhead
\(3 + 2\) & \texttt{3\ +\ 2} & 5 \\
\(3 - 2\) & \texttt{3\ -\ 2} & 1 \\
\(3 \cdot2\) & \texttt{3\ *\ 2} & 6 \\
\(3 / 2\) & \texttt{3\ /\ 2} & 1.5 \\
\bottomrule
\end{longtable}

\hypertarget{potux119gowanie}{%
\paragraph*{Potęgowanie}\label{potux119gowanie}}
\addcontentsline{toc}{paragraph}{Potęgowanie}

\begin{longtable}[]{@{}ccc@{}}
\toprule
Math & \texttt{R} code & Result \\
\midrule
\endhead
\(3^2\) & \texttt{3\ \^{}\ 2} & 9 \\
\(2^{(-3)}\) & \texttt{2\ \^{}\ (-3)} & 0.125 \\
\(100^{1/2}\) & \texttt{100\ \^{}\ (1\ /\ 2)} & 10 \\
\(\sqrt{100}\) & \texttt{sqrt(100)} & 10 \\
\bottomrule
\end{longtable}

\hypertarget{staux142e-matematyczne}{%
\paragraph*{Stałe matematyczne}\label{staux142e-matematyczne}}
\addcontentsline{toc}{paragraph}{Stałe matematyczne}

\begin{longtable}[]{@{}ccc@{}}
\toprule
Math & \texttt{R} code & Result \\
\midrule
\endhead
\(\pi\) & \texttt{pi} & 3.1415927 \\
\(e\) & \texttt{exp(1)} & 2.7182818 \\
\bottomrule
\end{longtable}

\hypertarget{logarytmy}{%
\paragraph*{Logarytmy}\label{logarytmy}}
\addcontentsline{toc}{paragraph}{Logarytmy}

Zauważ, że będziemy używać zamiennie \(\ln\) i \(\log\), aby oznaczać
logarytm naturalny. W \texttt{R} nie ma \texttt{ln()}, zamiast tego
używa \texttt{log()} do oznaczenia logarytmu naturalnego.

\begin{longtable}[]{@{}ccc@{}}
\toprule
Math & \texttt{R} code & Result \\
\midrule
\endhead
\(\log(e)\) & \texttt{log(exp(1))} & 1 \\
\(\log_{10}(1000)\) & \texttt{log10(1000)} & 3 \\
\(\log_{2}(8)\) & \texttt{log2(8)} & 3 \\
\(\log_{4}(16)\) & \texttt{log(16,\ base\ =\ 4)} & 2 \\
\bottomrule
\end{longtable}

\hypertarget{trigonometry}{%
\paragraph*{Trigonometry}\label{trigonometry}}
\addcontentsline{toc}{paragraph}{Trigonometry}

\begin{longtable}[]{@{}ccc@{}}
\toprule
Math & \texttt{R} code & Result \\
\midrule
\endhead
\(\sin(\pi / 2)\) & \texttt{sin(pi\ /\ 2)} & 1 \\
\(\cos(0)\) & \texttt{cos(0)} & 1 \\
\bottomrule
\end{longtable}

\hypertarget{jak-uzyskaux107-pomoc}{%
\subsection{Jak uzyskać pomoc?}\label{jak-uzyskaux107-pomoc}}

Używając \texttt{R} jako kalkulatora, widzieliśmy wiele funkcji:
\texttt{sqrt()}, \texttt{exp()}, \texttt{log()} i \texttt{sin()}. Aby
uzyskać dokumentację dotyczącą funkcji w \texttt{R}, po prostu wstaw
znak zapytania przed nazwą funkcji lub wywołaj funkcję
\texttt{help(funkcja)}, a RStudio wyświetli dokumentację, na przykład:

\begin{Shaded}
\begin{Highlighting}[]
\NormalTok{?log}
\NormalTok{?sin}
\NormalTok{?paste}
\NormalTok{?lm}
\FunctionTok{help}\NormalTok{(lm)   }\CommentTok{\# help() is equivalent}
\FunctionTok{help}\NormalTok{(ggplot,}\AttributeTok{package=}\StringTok{"ggplot2"}\NormalTok{)  }\CommentTok{\# show help from a certain package}
\end{Highlighting}
\end{Shaded}

Często jedną z najtrudniejszych rzeczy do zrobienia podczas nauki „R''
jest poproszenie o pomoc. Najpierw musisz zdecydować się poprosić o
pomoc, a następnie musisz wiedzieć \emph{jak} poprosić o pomoc. Pierwszą
linią obrony powinien być przekazany Google komunikat o błędzie lub
krótki opis problemu. (Umiejętność rozwiązywania problemów za pomocą tej
metody szybko staje się niezwykle cenną umiejętnością.) Jeśli to
zawiedzie, a w końcu tak się stanie, powinieneś poprosić o pomoc. Jest
wiele rzeczy, które powinieneś uwzględnić, kontaktując się z
instruktorem lub publikując post na stronie pomocy, takiej jak {[}Stack
Overflow{]} (\url{https://stackoverflow.com}).

\begin{itemize}
\tightlist
\item
  Opisz, czego oczekujesz od kodu.
\item
  Podaj ostateczny cel, który chcesz osiągnąć. (Czasami to, czego
  oczekujesz od kodu, nie jest tym, co chcesz faktycznie zrobić.)
\item
  Podaj pełny tekst wszelkich otrzymanych błędów. --- Podaj
  wystarczającą ilość kodu, aby odtworzyć błąd. Często na potrzeby tego
  kursu możesz po prostu opublikować cały skrypt \texttt{.R} lub
  \texttt{.Rmd} w \texttt{slack}.
\item
  Czasami pomocne jest również dołączenie zrzutu ekranu całego okna
  RStudio, gdy wystąpi błąd.
\end{itemize}

Jeśli wykonasz te kroki, problem zostanie rozwiązany znacznie szybciej,
a przy okazji prawdopodobnie dowiesz się więcej. Nie zniechęcaj się,
jeśli napotkasz błędy i trudności podczas nauki \texttt{R}. (Lub
jakakolwiek inna umiejętność techniczna.) Jest to po prostu część
procesu uczenia się.

\hypertarget{instalowanie-pakietuxf3w}{%
\subsection{Instalowanie pakietów}\label{instalowanie-pakietuxf3w}}

\texttt{R} ma wiele wbudowanych funkcji i zbiorów danych, ale jedną z
głównych zalet \texttt{R} jako projektu open source jest system
pakietów. Pakiety dodają dodatkowe funkcje i dane. Często, jeśli chcesz
coś zrobić w \texttt{R}, a nie jest to domyślnie dostępne, istnieje duża
szansa, że istnieje pakiet, który spełni Twoje potrzeby.

Aby zainstalować pakiet, użyj funkcji \texttt{install.packages()}.
Pomyśl o tym jak o zakupie książki kucharskiej w sklepie, zabraniu jej
do domu i umieszczeniu na półce (tj. w bibliotece):

\begin{Shaded}
\begin{Highlighting}[]
\FunctionTok{install.packages}\NormalTok{(}\StringTok{"ggplot2"}\NormalTok{)}
\end{Highlighting}
\end{Shaded}

Po zainstalowaniu pakietu należy go załadować do bieżącej sesji
\texttt{R}, zanim będzie można go użyć. Potraktuj to jak zdjęcie książki
z półki i otwarcie jej do przeczytania.

\begin{Shaded}
\begin{Highlighting}[]
\FunctionTok{library}\NormalTok{(ggplot2)}
\end{Highlighting}
\end{Shaded}

Po zamknięciu „R'' wszystkie opakowania zostaną zamknięte i odłożone na
wyimaginowaną półkę. Następnym razem, gdy otworzysz \texttt{R}, nie
musisz ponownie instalować pakietu, ale musisz załadować wszystkie
pakiety, których zamierzasz używać, wywołując \texttt{library()}.

\hypertarget{code-output}{%
\subsection{„Kod'' a dane wyjściowe w tym
dokumencie}\label{code-output}}

Wszystkie linie wyjściowe są poprzedzone znakiem „\#\#'', aby umożliwić
rozróżnienie. Typowy fragment kodu z danymi wyjściowymi będzie zatem
wyglądać następująco:

\begin{Shaded}
\begin{Highlighting}[]
\DecValTok{1} \SpecialCharTok{+} \DecValTok{3}
\end{Highlighting}
\end{Shaded}

\begin{verbatim}
## [1] 4
\end{verbatim}

\begin{Shaded}
\begin{Highlighting}[]
\CommentTok{\# wszystko po \# jest komentarzem i R ignoruje to.}
\end{Highlighting}
\end{Shaded}

gdzie w pierwszej linii widzisz kod \texttt{R}, a w drugiej linii wynik.
Jak wspomniano, linia ta zaczyna się od \texttt{\#\#} oznaczającego
\emph{to jest wyjście}, po którym następuje \texttt{{[}1{]}}
(wskazujące, że jest to wektor o długości \emph{jeden} - więcej na ten
temat poniżej!), po którym następuje rzeczywista wartość wynik -
\texttt{1\ +\ 3\ =\ 4}!

Zauważ, że możesz po prostu skopiować i wkleić cały widoczny kod do
konsoli „R''. Tak naprawdę jesteś \emph{zdecydowanie} zachęcany do
zrobienia tego i wypróbowania \textbf{całego kodu}, który widzisz w tej
książce.

Na koniec należy pamiętać, że ten sposób wyświetlania wyników jest w
pełni naszym wyborem w tym podręczniku i że innych formatów wyjściowych
należy spodziewać się gdzie indziej. Na przykład w mojej konsoli
„RStudio'' powyższy kod i dane wyjściowe wyglądają następująco:

\begin{Shaded}
\begin{Highlighting}[]
\SpecialCharTok{\textgreater{}} \DecValTok{1} \SpecialCharTok{+} \DecValTok{3}
\NormalTok{[}\DecValTok{1}\NormalTok{] }\DecValTok{4}
\end{Highlighting}
\end{Shaded}

\hypertarget{data-types}{%
\subsection{Typy danych}\label{data-types}}

\texttt{R} ma kilka podstawowych \emph{typów danych}. Chociaż \texttt{R}
nie jest \emph{językiem o silnie typowanym typie} (tj. w większości
przypadków możesz mieć agnostyka w stosunku do typów), warto wiedzieć,
jakie typy danych są dla ciebie dostępne:

\begin{itemize}
\tightlist
\item
  Numeric

  \begin{itemize}
  \tightlist
  \item
    również znany jako Double. Domyślny typ w przypadku liczb.
  \item
    Przykład: \texttt{1}, \texttt{1.0}, \texttt{42.5}
  \end{itemize}
\item
  Integer

  \begin{itemize}
  \tightlist
  \item
    Przykład: \texttt{1L}, \texttt{2L}, \texttt{42L}
  \end{itemize}
\item
  Complex

  \begin{itemize}
  \tightlist
  \item
    Przykład: \texttt{4\ +\ 2i}
  \end{itemize}
\item
  Logical

  \begin{itemize}
  \tightlist
  \item
    Dwie możliwe wartości: \texttt{TRUE} and \texttt{FALSE}
  \item
    Można również używać \texttt{T} oraz \texttt{F}, ale to \emph{nie}
    jest wskazane.
  \item
    \texttt{NA} jest również wartością logiczną.
  \end{itemize}
\item
  Character

  \begin{itemize}
  \tightlist
  \item
    Przykład: \texttt{"a"}, \texttt{"Statistics"},
    \texttt{"1\ plus\ 2."}
  \end{itemize}
\item
  Kategoryczne lub \texttt{factor}

  \begin{itemize}
  \tightlist
  \item
    Miks integer oraz character. Zmienna typu \texttt{factor} przypisuje
    etykietę do wartości liczbowej.
  \item
    Na przykład \texttt{factor(x=c(0,1),labels=c("male","female"))}
    przypisuje słowo \emph{male} do wartości liczbowych \texttt{0}, a
    słowo \emph{female} do wartości „1''.
  \end{itemize}
\end{itemize}

\hypertarget{struktury-danych}{%
\subsection{Struktury danych}\label{struktury-danych}}

\texttt{R} ma także kilka podstawowych \emph{struktur} danych. Struktura
danych jest albo jednorodna (wszystkie elementy są tego samego typu
danych), albo heterogeniczna (elementy mogą należeć do więcej niż
jednego typu danych).

\begin{longtable}[]{@{}ccc@{}}
\toprule
Wymiar & \textbf{Homogeniczna} & \textbf{Heterogeniczna} \\
\midrule
\endhead
1 & Vector & List \\
2 & Matrix & Data Frame \\
3+ & Array & nested Lists \\
\bottomrule
\end{longtable}

\hypertarget{wektory-vectors}{%
\subsubsection{Wektory (Vectors)}\label{wektory-vectors}}

Wiele operacji w \texttt{R} w dużym stopniu wykorzystuje
\textbf{wektory}. Wektor jest \emph{pojemnikiem} na obiekty identycznego
typu (patrz @ref(data-types) powyżej). Wektory w \texttt{R} są
indeksowane począwszy od \texttt{1}. To właśnie wskazuje „{[}1{]}'' na
wyjściu, że pierwszy element wyświetlanego wiersza jest pierwszym
elementem wektora. Większe wektory będą rozpoczynać dodatkowe wiersze od
czegoś w rodzaju „{[}7{]}'', gdzie „7'' jest indeksem pierwszego
elementu tego wiersza.

Prawdopodobnie najczęstszym sposobem tworzenia wektora w „R'' jest
użycie funkcji „c()'', która jest skrótem od „połącz'' (coerce). Jak
sama nazwa wskazuje, łączy w sobie listę elementów oddzielonych
przecinkami.

\begin{Shaded}
\begin{Highlighting}[]
\FunctionTok{c}\NormalTok{(}\DecValTok{1}\NormalTok{, }\DecValTok{3}\NormalTok{, }\DecValTok{5}\NormalTok{, }\DecValTok{7}\NormalTok{, }\DecValTok{8}\NormalTok{, }\DecValTok{9}\NormalTok{)}
\end{Highlighting}
\end{Shaded}

\begin{verbatim}
## [1] 1 3 5 7 8 9
\end{verbatim}

\begin{Shaded}
\begin{Highlighting}[]
\FunctionTok{c}\NormalTok{(}\DecValTok{5}\NormalTok{,}\DecValTok{5}\NormalTok{,}\DecValTok{5}\NormalTok{)}
\end{Highlighting}
\end{Shaded}

\begin{verbatim}
## [1] 5 5 5
\end{verbatim}

Tutaj „R'' po prostu wyprowadza ten wektor. Jeśli chcielibyśmy zapisać
ten wektor w \textbf{zmiennej}, możemy to zrobić za pomocą operatora
\textbf{przypisania} \texttt{=}. W tym przypadku zmienna „x''
przechowuje teraz właśnie utworzony wektor i możemy uzyskać dostęp do
wektora, wpisując „x''.

\begin{Shaded}
\begin{Highlighting}[]
\NormalTok{x }\OtherTok{=} \FunctionTok{c}\NormalTok{(}\DecValTok{1}\NormalTok{, }\DecValTok{3}\NormalTok{, }\DecValTok{5}\NormalTok{, }\DecValTok{7}\NormalTok{, }\DecValTok{8}\NormalTok{, }\DecValTok{9}\NormalTok{)}
\NormalTok{x}
\end{Highlighting}
\end{Shaded}

\begin{verbatim}
## [1] 1 3 5 7 8 9
\end{verbatim}

Na marginesie, operator przypisania w \texttt{R} ma długą historię,
częściowo za sprawą klawiszy dostępnych na
\href{https://twitter.com/kwbroman/\%20status/747829864091127809}{klawiaturach
twórców języka \texttt{S}.} (który poprzedzał \texttt{R}.) Dla
uproszczenia użyjemy \texttt{=}, ale pamiętaj, że często zobaczysz
\texttt{\textless{}-} jako operator przypisania.

Ponieważ wektory muszą zawierać elementy tego samego typu, „R''
automatycznie \textbf{wymusi} (tj. skonwertuje) na jeden typ podczas
próby utworzenia wektora łączącego wiele typów.

\begin{Shaded}
\begin{Highlighting}[]
\FunctionTok{c}\NormalTok{(}\DecValTok{42}\NormalTok{, }\StringTok{"Statistics"}\NormalTok{, }\ConstantTok{TRUE}\NormalTok{)}
\end{Highlighting}
\end{Shaded}

\begin{verbatim}
## [1] "42"         "Statistics" "TRUE"
\end{verbatim}

\begin{Shaded}
\begin{Highlighting}[]
\FunctionTok{c}\NormalTok{(}\DecValTok{42}\NormalTok{, }\ConstantTok{TRUE}\NormalTok{)}
\end{Highlighting}
\end{Shaded}

\begin{verbatim}
## [1] 42  1
\end{verbatim}

Często możesz chcieć utworzyć wektor w oparciu o sekwencję liczb.
Najszybciej i najłatwiej to zrobić za pomocą operatora \texttt{:}, który
tworzy sekwencję liczb całkowitych pomiędzy dwiema określonymi liczbami
całkowitymi.

\begin{Shaded}
\begin{Highlighting}[]
\NormalTok{(}\AttributeTok{y =} \DecValTok{1}\SpecialCharTok{:}\DecValTok{100}\NormalTok{)}
\end{Highlighting}
\end{Shaded}

\begin{verbatim}
##   [1]   1   2   3   4   5   6   7   8   9  10  11  12  13  14  15  16  17  18
##  [19]  19  20  21  22  23  24  25  26  27  28  29  30  31  32  33  34  35  36
##  [37]  37  38  39  40  41  42  43  44  45  46  47  48  49  50  51  52  53  54
##  [55]  55  56  57  58  59  60  61  62  63  64  65  66  67  68  69  70  71  72
##  [73]  73  74  75  76  77  78  79  80  81  82  83  84  85  86  87  88  89  90
##  [91]  91  92  93  94  95  96  97  98  99 100
\end{verbatim}

Tutaj widzimy „R'' oznaczający wiersze po pierwszym, ponieważ jest to
duży wektor. Widzimy także, że umieszczając przypisanie w nawiasach,
„R'' przechowuje wektor w zmiennej zwanej „y'' i automatycznie wysyła
„y'' do konsoli.

Zauważ, że skalary nie istnieją w \texttt{R}. Są to po prostu wektory o
długości „1''.

\begin{Shaded}
\begin{Highlighting}[]
\DecValTok{2}
\end{Highlighting}
\end{Shaded}

\begin{verbatim}
## [1] 2
\end{verbatim}

Jeśli chcemy utworzyć sekwencję, która nie jest ograniczona do liczb
całkowitych i zwiększa się o 1 na raz, możemy użyć funkcji
\texttt{seq()}.

\begin{Shaded}
\begin{Highlighting}[]
\FunctionTok{seq}\NormalTok{(}\AttributeTok{from =} \FloatTok{1.5}\NormalTok{, }\AttributeTok{to =} \FloatTok{4.2}\NormalTok{, }\AttributeTok{by =} \FloatTok{0.1}\NormalTok{)}
\end{Highlighting}
\end{Shaded}

\begin{verbatim}
##  [1] 1.5 1.6 1.7 1.8 1.9 2.0 2.1 2.2 2.3 2.4 2.5 2.6 2.7 2.8 2.9 3.0 3.1 3.2 3.3
## [20] 3.4 3.5 3.6 3.7 3.8 3.9 4.0 4.1 4.2
\end{verbatim}

Funkcje omówimy szczegółowo później, ale zwróć uwagę, że etykiety
wejściowe „od'', „do'' i „przez'' są opcjonalne.

\begin{Shaded}
\begin{Highlighting}[]
\FunctionTok{seq}\NormalTok{(}\FloatTok{1.5}\NormalTok{, }\FloatTok{4.2}\NormalTok{, }\FloatTok{0.1}\NormalTok{)}
\end{Highlighting}
\end{Shaded}

\begin{verbatim}
##  [1] 1.5 1.6 1.7 1.8 1.9 2.0 2.1 2.2 2.3 2.4 2.5 2.6 2.7 2.8 2.9 3.0 3.1 3.2 3.3
## [20] 3.4 3.5 3.6 3.7 3.8 3.9 4.0 4.1 4.2
\end{verbatim}

Another common operation to create a vector is \texttt{rep()}, which can
repeat a single value a number of times.

\begin{Shaded}
\begin{Highlighting}[]
\FunctionTok{rep}\NormalTok{(}\StringTok{"A"}\NormalTok{, }\AttributeTok{times =} \DecValTok{10}\NormalTok{)}
\end{Highlighting}
\end{Shaded}

\begin{verbatim}
##  [1] "A" "A" "A" "A" "A" "A" "A" "A" "A" "A"
\end{verbatim}

Funkcji „rep()'' można użyć do powtórzenia wektora określoną liczbę
razy.

\begin{Shaded}
\begin{Highlighting}[]
\FunctionTok{rep}\NormalTok{(x, }\AttributeTok{times =} \DecValTok{3}\NormalTok{)}
\end{Highlighting}
\end{Shaded}

\begin{verbatim}
##  [1] 1 3 5 7 8 9 1 3 5 7 8 9 1 3 5 7 8 9
\end{verbatim}

Widzieliśmy teraz cztery różne sposoby tworzenia wektorów: -
\texttt{c()} - \texttt{:} - \texttt{seq()} - \texttt{rep()} Do tej pory
używaliśmy ich głównie osobno, ale często używa się ich razem.

\begin{Shaded}
\begin{Highlighting}[]
\FunctionTok{c}\NormalTok{(x, }\FunctionTok{rep}\NormalTok{(}\FunctionTok{seq}\NormalTok{(}\DecValTok{1}\NormalTok{, }\DecValTok{9}\NormalTok{, }\DecValTok{2}\NormalTok{), }\DecValTok{3}\NormalTok{), }\FunctionTok{c}\NormalTok{(}\DecValTok{1}\NormalTok{, }\DecValTok{2}\NormalTok{, }\DecValTok{3}\NormalTok{), }\DecValTok{42}\NormalTok{, }\DecValTok{2}\SpecialCharTok{:}\DecValTok{4}\NormalTok{)}
\end{Highlighting}
\end{Shaded}

\begin{verbatim}
##  [1]  1  3  5  7  8  9  1  3  5  7  9  1  3  5  7  9  1  3  5  7  9  1  2  3 42
## [26]  2  3  4
\end{verbatim}

Długość wektora można uzyskać za pomocą funkcji „length()''.

\begin{Shaded}
\begin{Highlighting}[]
\FunctionTok{length}\NormalTok{(x)}
\end{Highlighting}
\end{Shaded}

\begin{verbatim}
## [1] 6
\end{verbatim}

\begin{Shaded}
\begin{Highlighting}[]
\FunctionTok{length}\NormalTok{(y)}
\end{Highlighting}
\end{Shaded}

\begin{verbatim}
## [1] 100
\end{verbatim}

\begin{warning}
Wypróbujmy to! \textbf{Twoja kolej}:
\end{warning}

\hypertarget{zadanie-1}{%
\paragraph{Zadanie 1}\label{zadanie-1}}

\begin{enumerate}
\def\labelenumi{\arabic{enumi}.}
\tightlist
\item
  Utwórz wektor pięciu jedynek, czyli: \texttt{{[}1,1,1,1,1{]}}
\item
  Zauważ, że operator dwukropka \texttt{a:b} jest skrótem od
  \emph{konstruuj sekwencję \textbf{od} \texttt{a} \textbf{do}
  \texttt{b}}. Utwórz wektor, od który będzie odliczał od 10 do 0, czyli
  będzie wyglądał tak: \texttt{{[}10,9,8,7,6,5,4,3,2,1,0{]}}!
\item
  funkcja \texttt{rep} pobiera dodatkowe argumenty \texttt{times} (jak
  powyżej) i \texttt{each}, które mówią, jak często \emph{każdy element}
  powinien być powtarzany (w przeciwieństwie do całego wektora
  wejściowego). Użyj \texttt{rep}, aby utworzyć wektor wyglądający tak:
  \texttt{{[}1\ 1\ 1\ 2\ 2\ 2\ 3\ 3\ 3\ 1\ 1\ 1\ 2\ 2\ 2\ 3\ 3\ 3{]}}
\end{enumerate}

\hypertarget{wywoux142ywanie-subsetting}{%
\paragraph{Wywoływanie (Subsetting)}\label{wywoux142ywanie-subsetting}}

Aby wywołać z wektora tylko niektóre jego elementy, używamy nawiasów
kwadratowych „{[}{]}''. Widzimy tutaj, że \texttt{x{[}1{]}} zwraca
pierwszy element, a \texttt{x{[}3{]}} zwraca trzeci element:

\begin{Shaded}
\begin{Highlighting}[]
\NormalTok{x}
\end{Highlighting}
\end{Shaded}

\begin{verbatim}
## [1] 1 3 5 7 8 9
\end{verbatim}

\begin{Shaded}
\begin{Highlighting}[]
\NormalTok{x[}\DecValTok{1}\NormalTok{]}
\end{Highlighting}
\end{Shaded}

\begin{verbatim}
## [1] 1
\end{verbatim}

\begin{Shaded}
\begin{Highlighting}[]
\NormalTok{x[}\DecValTok{3}\NormalTok{]}
\end{Highlighting}
\end{Shaded}

\begin{verbatim}
## [1] 5
\end{verbatim}

Możemy też wykluczyć pewne indeksy, w tym przypadku drugi element.

\begin{Shaded}
\begin{Highlighting}[]
\NormalTok{x[}\SpecialCharTok{{-}}\DecValTok{2}\NormalTok{]}
\end{Highlighting}
\end{Shaded}

\begin{verbatim}
## [1] 1 5 7 8 9
\end{verbatim}

Na koniec widzimy, że możemy dokonać podzbioru w oparciu o wektor
indeksów.

\begin{Shaded}
\begin{Highlighting}[]
\NormalTok{x[}\DecValTok{1}\SpecialCharTok{:}\DecValTok{3}\NormalTok{]}
\end{Highlighting}
\end{Shaded}

\begin{verbatim}
## [1] 1 3 5
\end{verbatim}

\begin{Shaded}
\begin{Highlighting}[]
\NormalTok{x[}\FunctionTok{c}\NormalTok{(}\DecValTok{1}\NormalTok{,}\DecValTok{3}\NormalTok{,}\DecValTok{4}\NormalTok{)]}
\end{Highlighting}
\end{Shaded}

\begin{verbatim}
## [1] 1 5 7
\end{verbatim}

Wszystkie powyższe stanowią podzbiór wektora za pomocą wektora indeksów.
(Pamiętaj, że pojedyncza liczba jest nadal wektorem.) Zamiast tego
moglibyśmy użyć wektora wartości logicznych.

\begin{Shaded}
\begin{Highlighting}[]
\NormalTok{z }\OtherTok{=} \FunctionTok{c}\NormalTok{(}\ConstantTok{TRUE}\NormalTok{, }\ConstantTok{TRUE}\NormalTok{, }\ConstantTok{FALSE}\NormalTok{, }\ConstantTok{TRUE}\NormalTok{, }\ConstantTok{TRUE}\NormalTok{, }\ConstantTok{FALSE}\NormalTok{)}
\NormalTok{z}
\end{Highlighting}
\end{Shaded}

\begin{verbatim}
## [1]  TRUE  TRUE FALSE  TRUE  TRUE FALSE
\end{verbatim}

\begin{Shaded}
\begin{Highlighting}[]
\NormalTok{x[z]}
\end{Highlighting}
\end{Shaded}

\begin{verbatim}
## [1] 1 3 7 8
\end{verbatim}

\texttt{R} może wykonywać wiele operacji zarówno na wektorach, jak i
skalarach:

\begin{Shaded}
\begin{Highlighting}[]
\NormalTok{x }\OtherTok{=} \DecValTok{1}\SpecialCharTok{:}\DecValTok{10}  \CommentTok{\# wektor}
\NormalTok{x }\SpecialCharTok{+} \DecValTok{1}     \CommentTok{\# dodawanie wektora i skalara}
\end{Highlighting}
\end{Shaded}

\begin{verbatim}
##  [1]  2  3  4  5  6  7  8  9 10 11
\end{verbatim}

\begin{Shaded}
\begin{Highlighting}[]
\DecValTok{2} \SpecialCharTok{*}\NormalTok{ x     }\CommentTok{\# pomnóż elementy wektora przez 2}
\end{Highlighting}
\end{Shaded}

\begin{verbatim}
##  [1]  2  4  6  8 10 12 14 16 18 20
\end{verbatim}

\begin{Shaded}
\begin{Highlighting}[]
\DecValTok{2} \SpecialCharTok{\^{}}\NormalTok{ x     }\CommentTok{\# weź 2 do potęgi elementów x}
\end{Highlighting}
\end{Shaded}

\begin{verbatim}
##  [1]    2    4    8   16   32   64  128  256  512 1024
\end{verbatim}

\begin{Shaded}
\begin{Highlighting}[]
\FunctionTok{sqrt}\NormalTok{(x)   }\CommentTok{\# oblicz pierwiastek elementów x}
\end{Highlighting}
\end{Shaded}

\begin{verbatim}
##  [1] 1.000000 1.414214 1.732051 2.000000 2.236068 2.449490 2.645751 2.828427
##  [9] 3.000000 3.162278
\end{verbatim}

\begin{Shaded}
\begin{Highlighting}[]
\FunctionTok{log}\NormalTok{(x)    }\CommentTok{\# logarytm naturalny elementów x}
\end{Highlighting}
\end{Shaded}

\begin{verbatim}
##  [1] 0.0000000 0.6931472 1.0986123 1.3862944 1.6094379 1.7917595 1.9459101
##  [8] 2.0794415 2.1972246 2.3025851
\end{verbatim}

\begin{Shaded}
\begin{Highlighting}[]
\NormalTok{x }\SpecialCharTok{+} \DecValTok{2}\SpecialCharTok{*}\NormalTok{x   }\CommentTok{\# dodaj wektor x do wektora 2x}
\end{Highlighting}
\end{Shaded}

\begin{verbatim}
##  [1]  3  6  9 12 15 18 21 24 27 30
\end{verbatim}

Widzimy, że gdy funkcja taka jak „log()'' jest wywoływana na wektorze
„x'', zwracany jest wektor, który zastosował tę funkcję do każdego
elementu wektora „x''.

\hypertarget{operatory-logiczne}{%
\subsubsection{Operatory logiczne}\label{operatory-logiczne}}

\begin{longtable}[]{@{}
  >{\raggedright\arraybackslash}p{(\columnwidth - 6\tabcolsep) * \real{0.1538}}
  >{\centering\arraybackslash}p{(\columnwidth - 6\tabcolsep) * \real{0.3538}}
  >{\centering\arraybackslash}p{(\columnwidth - 6\tabcolsep) * \real{0.3538}}
  >{\centering\arraybackslash}p{(\columnwidth - 6\tabcolsep) * \real{0.1385}}@{}}
\toprule
\begin{minipage}[b]{\linewidth}\raggedright
Operator
\end{minipage} & \begin{minipage}[b]{\linewidth}\centering
Podsumowanie
\end{minipage} & \begin{minipage}[b]{\linewidth}\centering
Przykład
\end{minipage} & \begin{minipage}[b]{\linewidth}\centering
Wynik
\end{minipage} \\
\midrule
\endhead
\texttt{x\ \textless{}\ y} & \texttt{x} mniejsze niż \texttt{y} &
\texttt{3\ \textless{}\ 42} & TRUE \\
\texttt{x\ \textgreater{}\ y} & \texttt{x} większe niż \texttt{y} &
\texttt{3\ \textgreater{}\ 42} & FALSE \\
\texttt{x\ \textless{}=\ y} & \texttt{x} mniejsze równe \texttt{y} &
\texttt{3\ \textless{}=\ 42} & TRUE \\
\texttt{x\ \textgreater{}=\ y} & \texttt{x} większe równe \texttt{y} &
\texttt{3\ \textgreater{}=\ 42} & FALSE \\
\texttt{x\ ==\ y} & \texttt{x} równe \texttt{y} & \texttt{3\ ==\ 42} &
FALSE \\
\texttt{x\ !=\ y} & \texttt{x} nie równe \texttt{y} & \texttt{3\ !=\ 42}
& TRUE \\
\texttt{!x} & nie \texttt{x} & \texttt{!(3\ \textgreater{}\ 42)} &
TRUE \\
\texttt{x\ \textbar{}\ y} & \texttt{x} lub \texttt{y} &
\texttt{(3\ \textgreater{}\ 42)\ \textbar{}\ TRUE} & TRUE \\
\texttt{x\ \&\ y} & \texttt{x} oraz \texttt{y} &
\texttt{(3\ \textless{}\ 4)\ \&\ (\ 42\ \textgreater{}\ 13)} & TRUE \\
\bottomrule
\end{longtable}

W \texttt{R} operatory logiczne działają także na wektorach:

\begin{Shaded}
\begin{Highlighting}[]
\NormalTok{x }\OtherTok{=} \FunctionTok{c}\NormalTok{(}\DecValTok{1}\NormalTok{, }\DecValTok{3}\NormalTok{, }\DecValTok{5}\NormalTok{, }\DecValTok{7}\NormalTok{, }\DecValTok{8}\NormalTok{, }\DecValTok{9}\NormalTok{)}
\end{Highlighting}
\end{Shaded}

\begin{Shaded}
\begin{Highlighting}[]
\NormalTok{x }\SpecialCharTok{\textgreater{}} \DecValTok{3}
\end{Highlighting}
\end{Shaded}

\begin{verbatim}
## [1] FALSE FALSE  TRUE  TRUE  TRUE  TRUE
\end{verbatim}

\begin{Shaded}
\begin{Highlighting}[]
\NormalTok{x }\SpecialCharTok{\textless{}} \DecValTok{3}
\end{Highlighting}
\end{Shaded}

\begin{verbatim}
## [1]  TRUE FALSE FALSE FALSE FALSE FALSE
\end{verbatim}

\begin{Shaded}
\begin{Highlighting}[]
\NormalTok{x }\SpecialCharTok{==} \DecValTok{3}
\end{Highlighting}
\end{Shaded}

\begin{verbatim}
## [1] FALSE  TRUE FALSE FALSE FALSE FALSE
\end{verbatim}

\begin{Shaded}
\begin{Highlighting}[]
\NormalTok{x }\SpecialCharTok{!=} \DecValTok{3}
\end{Highlighting}
\end{Shaded}

\begin{verbatim}
## [1]  TRUE FALSE  TRUE  TRUE  TRUE  TRUE
\end{verbatim}

\begin{Shaded}
\begin{Highlighting}[]
\NormalTok{x }\SpecialCharTok{==} \DecValTok{3} \SpecialCharTok{\&}\NormalTok{ x }\SpecialCharTok{!=} \DecValTok{3}
\end{Highlighting}
\end{Shaded}

\begin{verbatim}
## [1] FALSE FALSE FALSE FALSE FALSE FALSE
\end{verbatim}

\begin{Shaded}
\begin{Highlighting}[]
\NormalTok{x }\SpecialCharTok{==} \DecValTok{3} \SpecialCharTok{|}\NormalTok{ x }\SpecialCharTok{!=} \DecValTok{3}
\end{Highlighting}
\end{Shaded}

\begin{verbatim}
## [1] TRUE TRUE TRUE TRUE TRUE TRUE
\end{verbatim}

Jest to całkiem przydatne w przypadku wywoływania podzbiorów.

\begin{Shaded}
\begin{Highlighting}[]
\NormalTok{x[x }\SpecialCharTok{\textgreater{}} \DecValTok{3}\NormalTok{]}
\end{Highlighting}
\end{Shaded}

\begin{verbatim}
## [1] 5 7 8 9
\end{verbatim}

\begin{Shaded}
\begin{Highlighting}[]
\NormalTok{x[x }\SpecialCharTok{!=} \DecValTok{3}\NormalTok{]}
\end{Highlighting}
\end{Shaded}

\begin{verbatim}
## [1] 1 5 7 8 9
\end{verbatim}

\begin{Shaded}
\begin{Highlighting}[]
\FunctionTok{sum}\NormalTok{(x }\SpecialCharTok{\textgreater{}} \DecValTok{3}\NormalTok{)}
\end{Highlighting}
\end{Shaded}

\begin{verbatim}
## [1] 4
\end{verbatim}

\begin{Shaded}
\begin{Highlighting}[]
\FunctionTok{as.numeric}\NormalTok{(x }\SpecialCharTok{\textgreater{}} \DecValTok{3}\NormalTok{)}
\end{Highlighting}
\end{Shaded}

\begin{verbatim}
## [1] 0 0 1 1 1 1
\end{verbatim}

Tutaj widzieliśmy, że użycie funkcji \texttt{suma()} na wektorze
wartości logicznych \texttt{TRUE} i \texttt{FALSE}, będącym wynikiem
\texttt{x\ \textgreater{}\ 3} daje wynik numeryczny: właśnie
\emph{policzyłeś}, ile elementów z \texttt{x}, dla warunku
\texttt{\textgreater{}\ 3} ma wartość \texttt{TRUE}. Podczas wywołania
\texttt{sum()}, \texttt{R} najpierw automatycznie zamienia wartość
logiczną na numeryczną, gdzie \texttt{TRUE} to \texttt{1}, a
\texttt{FALSE} to \texttt{0}. To wymuszenie przejścia z logicznego na
numeryczne ma miejsce w przypadku większości operacji matematycznych.

\begin{Shaded}
\begin{Highlighting}[]
\CommentTok{\# which(warunek na x) zwraca true/false  }
\CommentTok{\# każdy indeks x warunke jest spełniony}
\FunctionTok{which}\NormalTok{(x }\SpecialCharTok{\textgreater{}} \DecValTok{3}\NormalTok{)}
\end{Highlighting}
\end{Shaded}

\begin{verbatim}
## [1] 3 4 5 6
\end{verbatim}

\begin{Shaded}
\begin{Highlighting}[]
\NormalTok{x[}\FunctionTok{which}\NormalTok{(x }\SpecialCharTok{\textgreater{}} \DecValTok{3}\NormalTok{)]}
\end{Highlighting}
\end{Shaded}

\begin{verbatim}
## [1] 5 7 8 9
\end{verbatim}

\begin{Shaded}
\begin{Highlighting}[]
\FunctionTok{max}\NormalTok{(x)}
\end{Highlighting}
\end{Shaded}

\begin{verbatim}
## [1] 9
\end{verbatim}

\begin{Shaded}
\begin{Highlighting}[]
\FunctionTok{which}\NormalTok{(x }\SpecialCharTok{==} \FunctionTok{max}\NormalTok{(x))}
\end{Highlighting}
\end{Shaded}

\begin{verbatim}
## [1] 6
\end{verbatim}

\begin{Shaded}
\begin{Highlighting}[]
\FunctionTok{which.max}\NormalTok{(x)}
\end{Highlighting}
\end{Shaded}

\begin{verbatim}
## [1] 6
\end{verbatim}

\hypertarget{zadanie-2}{%
\paragraph{Zadanie 2}\label{zadanie-2}}

\begin{enumerate}
\def\labelenumi{\arabic{enumi}.}
\tightlist
\item
  Utwórz wektor wypełniony 10 liczbami wylosowanymi z rozkładu
  jednostajnego (podpowiedź: użyj funkcji „runif'') i zapisz je w
  \texttt{x}.
\item
  Używając podzbioru logicznego jak powyżej, zbierz wszystkie elementy
  \texttt{x}, które są większe niż 0,5 i zapisz je w \texttt{y}.
\item
  używając funkcji „które'', zapisz \emph{indeksy} wszystkich elementów
  „x'', które są większe niż 0,5 w „iy''.
\item
  Sprawdź, czy „y'' i „x{[}iy{]}'' są identyczne.
\end{enumerate}

\hypertarget{macierze}{%
\subsubsection{Macierze}\label{macierze}}

„R'' może być również użyte do obliczeń \textbf{macierzowych}. Macierze
mają wiersze i kolumny zawierające jeden typ danych. W macierzy ważna
jest kolejność wierszy i kolumn. (Nie dotyczy to \emph{ramek danych}, co
zobaczymy później.)

Macierze można tworzyć za pomocą funkcji \texttt{matrix}.

\begin{Shaded}
\begin{Highlighting}[]
\NormalTok{x }\OtherTok{=} \DecValTok{1}\SpecialCharTok{:}\DecValTok{9}
\NormalTok{x}
\end{Highlighting}
\end{Shaded}

\begin{verbatim}
## [1] 1 2 3 4 5 6 7 8 9
\end{verbatim}

\begin{Shaded}
\begin{Highlighting}[]
\NormalTok{X }\OtherTok{=} \FunctionTok{matrix}\NormalTok{(x, }\AttributeTok{nrow =} \DecValTok{3}\NormalTok{, }\AttributeTok{ncol =} \DecValTok{3}\NormalTok{)}
\NormalTok{X}
\end{Highlighting}
\end{Shaded}

\begin{verbatim}
##      [,1] [,2] [,3]
## [1,]    1    4    7
## [2,]    2    5    8
## [3,]    3    6    9
\end{verbatim}

Zauważ, że w \texttt{R} rozróżniana jest wielkość liter (\texttt{x} vs
\texttt{X}).

Domyślnie funkcja \texttt{matrix} wypełnia dane w macierzy kolumna po
kolumnie. Ale możemy też powiedzieć „R'', aby zamiast tego wypełniał
wiersze:

\begin{Shaded}
\begin{Highlighting}[]
\NormalTok{Y }\OtherTok{=} \FunctionTok{matrix}\NormalTok{(x, }\AttributeTok{nrow =} \DecValTok{3}\NormalTok{, }\AttributeTok{ncol =} \DecValTok{3}\NormalTok{, }\AttributeTok{byrow =} \ConstantTok{TRUE}\NormalTok{)}
\NormalTok{Y}
\end{Highlighting}
\end{Shaded}

\begin{verbatim}
##      [,1] [,2] [,3]
## [1,]    1    2    3
## [2,]    4    5    6
## [3,]    7    8    9
\end{verbatim}

Możemy również utworzyć macierz o określonym wymiarze, w której każdy
element jest taki sam, w tym przypadku „0''.

\begin{Shaded}
\begin{Highlighting}[]
\NormalTok{Z }\OtherTok{=} \FunctionTok{matrix}\NormalTok{(}\DecValTok{0}\NormalTok{, }\DecValTok{2}\NormalTok{, }\DecValTok{4}\NormalTok{)}
\NormalTok{Z}
\end{Highlighting}
\end{Shaded}

\begin{verbatim}
##      [,1] [,2] [,3] [,4]
## [1,]    0    0    0    0
## [2,]    0    0    0    0
\end{verbatim}

Podobnie jak wektory, macierze można wywoływać za pomocą nawiasów
kwadratowych „{[}{]}''. Ponieważ jednak macierze są dwuwymiarowe,
podczas wywoływania musimy określić zarówno wiersz, jak i kolumnę.

\begin{Shaded}
\begin{Highlighting}[]
\NormalTok{X}
\end{Highlighting}
\end{Shaded}

\begin{verbatim}
##      [,1] [,2] [,3]
## [1,]    1    4    7
## [2,]    2    5    8
## [3,]    3    6    9
\end{verbatim}

\begin{Shaded}
\begin{Highlighting}[]
\NormalTok{X[}\DecValTok{1}\NormalTok{, }\DecValTok{2}\NormalTok{]}
\end{Highlighting}
\end{Shaded}

\begin{verbatim}
## [1] 4
\end{verbatim}

Tutaj uzyskaliśmy dostęp do elementu w pierwszym wierszu i drugiej
kolumnie. Możemy także podzielić cały wiersz lub kolumnę.

\begin{Shaded}
\begin{Highlighting}[]
\NormalTok{X[}\DecValTok{1}\NormalTok{, ]}
\end{Highlighting}
\end{Shaded}

\begin{verbatim}
## [1] 1 4 7
\end{verbatim}

\begin{Shaded}
\begin{Highlighting}[]
\NormalTok{X[, }\DecValTok{2}\NormalTok{]}
\end{Highlighting}
\end{Shaded}

\begin{verbatim}
## [1] 4 5 6
\end{verbatim}

Możemy również użyć wektorów do podzbioru więcej niż jednego wiersza lub
kolumny na raz. Tutaj podstawiamy pierwszą i trzecią kolumnę drugiego
wiersza:

\begin{Shaded}
\begin{Highlighting}[]
\NormalTok{X[}\DecValTok{2}\NormalTok{, }\FunctionTok{c}\NormalTok{(}\DecValTok{1}\NormalTok{, }\DecValTok{3}\NormalTok{)]}
\end{Highlighting}
\end{Shaded}

\begin{verbatim}
## [1] 2 8
\end{verbatim}

Macierze można również tworzyć łącząc wektory jako kolumny, używając
polecenia „cbind'', lub łącząc wektory jako wiersze, używając polecenia
„rbind''.

\begin{Shaded}
\begin{Highlighting}[]
\NormalTok{x }\OtherTok{=} \DecValTok{1}\SpecialCharTok{:}\DecValTok{9}
\FunctionTok{rev}\NormalTok{(x)}
\end{Highlighting}
\end{Shaded}

\begin{verbatim}
## [1] 9 8 7 6 5 4 3 2 1
\end{verbatim}

\begin{Shaded}
\begin{Highlighting}[]
\FunctionTok{rep}\NormalTok{(}\DecValTok{1}\NormalTok{, }\DecValTok{9}\NormalTok{)}
\end{Highlighting}
\end{Shaded}

\begin{verbatim}
## [1] 1 1 1 1 1 1 1 1 1
\end{verbatim}

\begin{Shaded}
\begin{Highlighting}[]
\FunctionTok{rbind}\NormalTok{(x, }\FunctionTok{rev}\NormalTok{(x), }\FunctionTok{rep}\NormalTok{(}\DecValTok{1}\NormalTok{, }\DecValTok{9}\NormalTok{))}
\end{Highlighting}
\end{Shaded}

\begin{verbatim}
##   [,1] [,2] [,3] [,4] [,5] [,6] [,7] [,8] [,9]
## x    1    2    3    4    5    6    7    8    9
##      9    8    7    6    5    4    3    2    1
##      1    1    1    1    1    1    1    1    1
\end{verbatim}

\begin{Shaded}
\begin{Highlighting}[]
\FunctionTok{cbind}\NormalTok{(}\AttributeTok{col\_1 =}\NormalTok{ x, }\AttributeTok{col\_2 =} \FunctionTok{rev}\NormalTok{(x), }\AttributeTok{col\_3 =} \FunctionTok{rep}\NormalTok{(}\DecValTok{1}\NormalTok{, }\DecValTok{9}\NormalTok{))}
\end{Highlighting}
\end{Shaded}

\begin{verbatim}
##       col_1 col_2 col_3
##  [1,]     1     9     1
##  [2,]     2     8     1
##  [3,]     3     7     1
##  [4,]     4     6     1
##  [5,]     5     5     1
##  [6,]     6     4     1
##  [7,]     7     3     1
##  [8,]     8     2     1
##  [9,]     9     1     1
\end{verbatim}

Używając \texttt{rbind} i \texttt{cbind}, możesz określić nazwy
„argumentów'', które będą używane jako nazwy kolumn.

„R'' można następnie wykorzystać do wykonania obliczeń macierzowych.

\begin{Shaded}
\begin{Highlighting}[]
\NormalTok{x }\OtherTok{=} \DecValTok{1}\SpecialCharTok{:}\DecValTok{9}
\NormalTok{y }\OtherTok{=} \DecValTok{9}\SpecialCharTok{:}\DecValTok{1}
\NormalTok{X }\OtherTok{=} \FunctionTok{matrix}\NormalTok{(x, }\DecValTok{3}\NormalTok{, }\DecValTok{3}\NormalTok{)}
\NormalTok{Y }\OtherTok{=} \FunctionTok{matrix}\NormalTok{(y, }\DecValTok{3}\NormalTok{, }\DecValTok{3}\NormalTok{)}
\NormalTok{X}
\end{Highlighting}
\end{Shaded}

\begin{verbatim}
##      [,1] [,2] [,3]
## [1,]    1    4    7
## [2,]    2    5    8
## [3,]    3    6    9
\end{verbatim}

\begin{Shaded}
\begin{Highlighting}[]
\NormalTok{Y}
\end{Highlighting}
\end{Shaded}

\begin{verbatim}
##      [,1] [,2] [,3]
## [1,]    9    6    3
## [2,]    8    5    2
## [3,]    7    4    1
\end{verbatim}

\begin{Shaded}
\begin{Highlighting}[]
\NormalTok{X }\SpecialCharTok{+}\NormalTok{ Y}
\end{Highlighting}
\end{Shaded}

\begin{verbatim}
##      [,1] [,2] [,3]
## [1,]   10   10   10
## [2,]   10   10   10
## [3,]   10   10   10
\end{verbatim}

\begin{Shaded}
\begin{Highlighting}[]
\NormalTok{X }\SpecialCharTok{{-}}\NormalTok{ Y}
\end{Highlighting}
\end{Shaded}

\begin{verbatim}
##      [,1] [,2] [,3]
## [1,]   -8   -2    4
## [2,]   -6    0    6
## [3,]   -4    2    8
\end{verbatim}

\begin{Shaded}
\begin{Highlighting}[]
\NormalTok{X }\SpecialCharTok{*}\NormalTok{ Y}
\end{Highlighting}
\end{Shaded}

\begin{verbatim}
##      [,1] [,2] [,3]
## [1,]    9   24   21
## [2,]   16   25   16
## [3,]   21   24    9
\end{verbatim}

\begin{Shaded}
\begin{Highlighting}[]
\NormalTok{X }\SpecialCharTok{/}\NormalTok{ Y}
\end{Highlighting}
\end{Shaded}

\begin{verbatim}
##           [,1]      [,2]     [,3]
## [1,] 0.1111111 0.6666667 2.333333
## [2,] 0.2500000 1.0000000 4.000000
## [3,] 0.4285714 1.5000000 9.000000
\end{verbatim}

Zauważ, że \texttt{X\ *\ Y} to \textbf{nie} mnożenie macierzy. Jest to
mnożenie \emph{element po elemencie}. (To samo dla \texttt{X\ /\ Y}).
Mnożenie macierzy wykorzystuje operator \texttt{\%*\%}. Inne funkcje
macierzowe obejmują „t()'', która daje transpozycję macierzy i
„solve()'', która zwraca odwrotność macierzy kwadratowej, jeśli jest ona
odwracalna.

\begin{Shaded}
\begin{Highlighting}[]
\NormalTok{X }\SpecialCharTok{\%*\%}\NormalTok{ Y}
\end{Highlighting}
\end{Shaded}

\begin{verbatim}
##      [,1] [,2] [,3]
## [1,]   90   54   18
## [2,]  114   69   24
## [3,]  138   84   30
\end{verbatim}

\begin{Shaded}
\begin{Highlighting}[]
\FunctionTok{t}\NormalTok{(X)}
\end{Highlighting}
\end{Shaded}

\begin{verbatim}
##      [,1] [,2] [,3]
## [1,]    1    2    3
## [2,]    4    5    6
## [3,]    7    8    9
\end{verbatim}

\hypertarget{arrays---tablice}{%
\subsubsection{Arrays - tablice}\label{arrays---tablice}}

Wektor jest tablicą jednowymiarową. Macierz jest tablicą dwuwymiarową. W
\texttt{R} możesz tworzyć tablice o dowolnej wymiarowości \texttt{N}.
Oto jak:

\begin{Shaded}
\begin{Highlighting}[]
\NormalTok{d }\OtherTok{=} \DecValTok{1}\SpecialCharTok{:}\DecValTok{16}
\NormalTok{d3 }\OtherTok{=} \FunctionTok{array}\NormalTok{(}\AttributeTok{data =}\NormalTok{ d,}\AttributeTok{dim =} \FunctionTok{c}\NormalTok{(}\DecValTok{4}\NormalTok{,}\DecValTok{2}\NormalTok{,}\DecValTok{2}\NormalTok{))}
\NormalTok{d4 }\OtherTok{=} \FunctionTok{array}\NormalTok{(}\AttributeTok{data =}\NormalTok{ d,}\AttributeTok{dim =} \FunctionTok{c}\NormalTok{(}\DecValTok{4}\NormalTok{,}\DecValTok{2}\NormalTok{,}\DecValTok{2}\NormalTok{,}\DecValTok{3}\NormalTok{)) }
\NormalTok{d3}
\end{Highlighting}
\end{Shaded}

\begin{verbatim}
## , , 1
## 
##      [,1] [,2]
## [1,]    1    5
## [2,]    2    6
## [3,]    3    7
## [4,]    4    8
## 
## , , 2
## 
##      [,1] [,2]
## [1,]    9   13
## [2,]   10   14
## [3,]   11   15
## [4,]   12   16
\end{verbatim}

Możesz zobaczyć, że \texttt{d3} to po prostu \emph{dwie} (4,2) macierze
ułożone jedna na drugiej, tak jakby były \emph{dwie strony}. Podobnie
„d4'' miałoby dwie strony i kolejne 3 rejestry w czwartym wymiarze. I
tak dalej. Możesz podzielić tablicę na części, tak jak wektor lub
macierz, pamiętając o indeksowaniu każdego wymiaru:

\begin{Shaded}
\begin{Highlighting}[]
\NormalTok{d3[ ,}\DecValTok{1}\NormalTok{,}\DecValTok{1}\NormalTok{]  }\CommentTok{\# wszystkie elementy z kolumny 1, tablica 1}
\end{Highlighting}
\end{Shaded}

\begin{verbatim}
## [1] 1 2 3 4
\end{verbatim}

\begin{Shaded}
\begin{Highlighting}[]
\NormalTok{d3[}\DecValTok{2}\SpecialCharTok{:}\DecValTok{3}\NormalTok{, , ]  }\CommentTok{\# rzędy 2:3 ze wszystkich tablic}
\end{Highlighting}
\end{Shaded}

\begin{verbatim}
## , , 1
## 
##      [,1] [,2]
## [1,]    2    6
## [2,]    3    7
## 
## , , 2
## 
##      [,1] [,2]
## [1,]   10   14
## [2,]   11   15
\end{verbatim}

\begin{Shaded}
\begin{Highlighting}[]
\NormalTok{d3[}\DecValTok{2}\NormalTok{,}\DecValTok{2}\NormalTok{, ]  }\CommentTok{\# rząd 2, kolumna 2 z wszystkich tablic.}
\end{Highlighting}
\end{Shaded}

\begin{verbatim}
## [1]  6 14
\end{verbatim}

\hypertarget{zadanie-3}{%
\paragraph{Zadanie 3}\label{zadanie-3}}

\begin{enumerate}
\def\labelenumi{\arabic{enumi}.}
\tightlist
\item
  Utwórz wektor zawierający „1,2,3,4,5'' o nazwie v.
\item
  Utwórz macierz (2,5) \texttt{m} zawierającą dane
  \texttt{1,2,3,4,5,6,7,8,9,10}. Pierwszym wierszem powinno być
  „1,2,3,4,5''.
\item
  Wykonaj mnożenie macierzy \texttt{m} przez \texttt{v}. Użyj polecenia
  \texttt{\%*\%}. Jaki wymiar ma wyjście?
\item
  Dlaczego \texttt{v\ \%*\%\ m} nie działa?
\end{enumerate}

\hypertarget{listy}{%
\subsubsection{Listy}\label{listy}}

Lista to jednowymiarowa \emph{heterogeniczna} struktura danych. Jest
więc indeksowany jak wektor z pojedynczą wartością całkowitą (lub z
nazwą), ale każdy element może zawierać element dowolnego typu. Wiele
struktur i wyników „R'' jest listami. Listy to niezwykle przydatne i
wszechstronne obiekty, dlatego upewnij się, że rozumiesz ich
zastosowanie:

\begin{Shaded}
\begin{Highlighting}[]
\CommentTok{\# creation without fieldnames}
\FunctionTok{list}\NormalTok{(}\DecValTok{42}\NormalTok{, }\StringTok{"Hello"}\NormalTok{, }\ConstantTok{TRUE}\NormalTok{)}
\end{Highlighting}
\end{Shaded}

\begin{verbatim}
## [[1]]
## [1] 42
## 
## [[2]]
## [1] "Hello"
## 
## [[3]]
## [1] TRUE
\end{verbatim}

\begin{Shaded}
\begin{Highlighting}[]
\CommentTok{\# creation with fieldnames}
\NormalTok{ex\_list }\OtherTok{=} \FunctionTok{list}\NormalTok{(}
  \AttributeTok{a =} \FunctionTok{c}\NormalTok{(}\DecValTok{1}\NormalTok{, }\DecValTok{2}\NormalTok{, }\DecValTok{3}\NormalTok{, }\DecValTok{4}\NormalTok{),}
  \AttributeTok{b =} \ConstantTok{TRUE}\NormalTok{,}
  \AttributeTok{c =} \StringTok{"Hello!"}\NormalTok{,}
  \AttributeTok{d =} \ControlFlowTok{function}\NormalTok{(}\AttributeTok{arg =} \DecValTok{42}\NormalTok{) \{}\FunctionTok{print}\NormalTok{(}\StringTok{"Hello World!"}\NormalTok{)\},}
  \AttributeTok{e =} \FunctionTok{diag}\NormalTok{(}\DecValTok{5}\NormalTok{)}
\NormalTok{)}
\end{Highlighting}
\end{Shaded}

Listy można wywoływać przy użyciu dwóch składni: operatora \texttt{\$} i
nawiasów kwadratowych \texttt{{[}{]}}. Operator \texttt{\$} zwraca
nazwany \textbf{element} listy. Składnia \texttt{{[}{]}} zwraca
\textbf{listę}, podczas gdy \texttt{{[}{[}{]}{]}} zwraca
\textbf{element} listy.

\begin{itemize}
\tightlist
\item
  \texttt{ex\_list{[}1{]}} zwraca listę zawierającą pierwszy element.
\item
  \texttt{ex\_list{[}{[}1{]}{]}} zwraca pierwszy element listy, w tym
  przypadku wektor.
\end{itemize}

\begin{Shaded}
\begin{Highlighting}[]
\CommentTok{\# wywoływanie}
\NormalTok{ex\_list}\SpecialCharTok{$}\NormalTok{e}
\end{Highlighting}
\end{Shaded}

\begin{verbatim}
##      [,1] [,2] [,3] [,4] [,5]
## [1,]    1    0    0    0    0
## [2,]    0    1    0    0    0
## [3,]    0    0    1    0    0
## [4,]    0    0    0    1    0
## [5,]    0    0    0    0    1
\end{verbatim}

\begin{Shaded}
\begin{Highlighting}[]
\NormalTok{ex\_list[}\DecValTok{1}\SpecialCharTok{:}\DecValTok{2}\NormalTok{]}
\end{Highlighting}
\end{Shaded}

\begin{verbatim}
## $a
## [1] 1 2 3 4
## 
## $b
## [1] TRUE
\end{verbatim}

\begin{Shaded}
\begin{Highlighting}[]
\NormalTok{ex\_list[}\DecValTok{1}\NormalTok{]}
\end{Highlighting}
\end{Shaded}

\begin{verbatim}
## $a
## [1] 1 2 3 4
\end{verbatim}

\begin{Shaded}
\begin{Highlighting}[]
\NormalTok{ex\_list[[}\DecValTok{1}\NormalTok{]]}
\end{Highlighting}
\end{Shaded}

\begin{verbatim}
## [1] 1 2 3 4
\end{verbatim}

\begin{Shaded}
\begin{Highlighting}[]
\NormalTok{ex\_list[}\FunctionTok{c}\NormalTok{(}\StringTok{"e"}\NormalTok{, }\StringTok{"a"}\NormalTok{)]}
\end{Highlighting}
\end{Shaded}

\begin{verbatim}
## $e
##      [,1] [,2] [,3] [,4] [,5]
## [1,]    1    0    0    0    0
## [2,]    0    1    0    0    0
## [3,]    0    0    1    0    0
## [4,]    0    0    0    1    0
## [5,]    0    0    0    0    1
## 
## $a
## [1] 1 2 3 4
\end{verbatim}

\begin{Shaded}
\begin{Highlighting}[]
\NormalTok{ex\_list[}\StringTok{"e"}\NormalTok{]}
\end{Highlighting}
\end{Shaded}

\begin{verbatim}
## $e
##      [,1] [,2] [,3] [,4] [,5]
## [1,]    1    0    0    0    0
## [2,]    0    1    0    0    0
## [3,]    0    0    1    0    0
## [4,]    0    0    0    1    0
## [5,]    0    0    0    0    1
\end{verbatim}

\begin{Shaded}
\begin{Highlighting}[]
\NormalTok{ex\_list[[}\StringTok{"e"}\NormalTok{]]}
\end{Highlighting}
\end{Shaded}

\begin{verbatim}
##      [,1] [,2] [,3] [,4] [,5]
## [1,]    1    0    0    0    0
## [2,]    0    1    0    0    0
## [3,]    0    0    1    0    0
## [4,]    0    0    0    1    0
## [5,]    0    0    0    0    1
\end{verbatim}

\begin{Shaded}
\begin{Highlighting}[]
\NormalTok{ex\_list}\SpecialCharTok{$}\NormalTok{d}
\end{Highlighting}
\end{Shaded}

\begin{verbatim}
## function(arg = 42) {print("Hello World!")}
\end{verbatim}

\begin{Shaded}
\begin{Highlighting}[]
\NormalTok{ex\_list}\SpecialCharTok{$}\FunctionTok{d}\NormalTok{(}\AttributeTok{arg =} \DecValTok{1}\NormalTok{)}
\end{Highlighting}
\end{Shaded}

\begin{verbatim}
## [1] "Hello World!"
\end{verbatim}

\hypertarget{zadanie-4}{%
\paragraph{Zadanie 4}\label{zadanie-4}}

\begin{enumerate}
\def\labelenumi{\arabic{enumi}.}
\tightlist
\item
  Skopiuj i wklej powyższy kod z \texttt{ex\_list} do swojej sesji R.
  Pamiętaj, że \texttt{lista} może zawierać dowolny obiekt \texttt{R}.
  Jak\ldots{} kolejna lista! Utwórz więc nową listę „nowa\_lista'',
  która ma dwa pola: pierwsze pole o nazwie „this'' z zawartością „jest
  niesamowite'' i drugie pole o nazwie „ex\_list'', które zawiera
  „ex\_list''.
\item
  Dostęp do elementów odbywa się jak na zwykłej liście, tylko z kilkoma
  warstwami. Pobierz element \texttt{c} z \texttt{ex\_list} w
  \texttt{nowa\_lista}!
\item
  Utwórz nowy wyraz z pierwszego elementu w
  \texttt{nowa\_lista\textquotesingle{},\ elementu\ pod\ etykietą}this`.
  Użyj funkcji „paste'', aby wydrukować na ekranie „R jest
  niesamowity''.
\end{enumerate}

\hypertarget{dataframes}{%
\subsection{Ramki Danych - Data Frames}\label{dataframes}}

Kiedy wprowadziliśmy „R'', widzieliśmy już wektory i macierze do
przechowywania danych. W tym kursie przedstawimy teraz \textbf{ramkę
danych}, która będzie najpopularniejszym sposobem przechowywania danych
i interakcji z nimi. „Data.frame'' jest podobna do „pandas.dataframe'' w
Pythonie lub Julii „DataFrame''. (Ale wersja \texttt{R} była pierwsza! )

\begin{Shaded}
\begin{Highlighting}[]
\NormalTok{example\_data }\OtherTok{=} \FunctionTok{data.frame}\NormalTok{(}\AttributeTok{x =} \FunctionTok{c}\NormalTok{(}\DecValTok{1}\NormalTok{, }\DecValTok{3}\NormalTok{, }\DecValTok{5}\NormalTok{, }\DecValTok{7}\NormalTok{, }\DecValTok{9}\NormalTok{, }\DecValTok{1}\NormalTok{, }\DecValTok{3}\NormalTok{, }\DecValTok{5}\NormalTok{, }\DecValTok{7}\NormalTok{, }\DecValTok{9}\NormalTok{),}
                          \AttributeTok{y =} \FunctionTok{c}\NormalTok{(}\FunctionTok{rep}\NormalTok{(}\StringTok{"Hello"}\NormalTok{, }\DecValTok{9}\NormalTok{), }\StringTok{"Goodbye"}\NormalTok{),}
                          \AttributeTok{z =} \FunctionTok{rep}\NormalTok{(}\FunctionTok{c}\NormalTok{(}\ConstantTok{TRUE}\NormalTok{, }\ConstantTok{FALSE}\NormalTok{), }\DecValTok{5}\NormalTok{))}
\end{Highlighting}
\end{Shaded}

W przeciwieństwie do macierzy, którą można traktować jako wektor
podzielony na wiersze i kolumny, ramka danych nie musi mieć tego samego
typu danych dla każdego elementu. Ramka danych to \textbf{lista}
wektorów, a każdy wektor ma \emph{nazwę}. Zatem każdy wektor musi
zawierać ten sam typ danych, ale różne wektory mogą przechowywać różne
typy danych. Należy jednak pamiętać, że wszystkie wektory muszą mieć
\textbf{tę samą długość} (w odróżnieniu od \texttt{listy})!

\begin{tip}
\textbf{data.frame} jest podobna do typowego arkusza kalkulacyjnego. Są
\emph{wiersze} i \emph{kolumny}. Wiersz jest zwykle uważany za
\emph{obserwację}, a każda kolumna jest określoną \emph{zmienną},
\emph{cechą} lub \emph{cechą} tej obserwacji.
\end{tip}

Spójrzmy na ramkę danych, którą właśnie utworzyliśmy powyżej:

\begin{Shaded}
\begin{Highlighting}[]
\NormalTok{example\_data}
\end{Highlighting}
\end{Shaded}

\begin{verbatim}
##    x       y     z
## 1  1   Hello  TRUE
## 2  3   Hello FALSE
## 3  5   Hello  TRUE
## 4  7   Hello FALSE
## 5  9   Hello  TRUE
## 6  1   Hello FALSE
## 7  3   Hello  TRUE
## 8  5   Hello FALSE
## 9  7   Hello  TRUE
## 10 9 Goodbye FALSE
\end{verbatim}

W przeciwieństwie do listy, która ma większą elastyczność, wszystkie
elementy ramki danych muszą być wektorami. Ponownie uzyskujemy dostęp do
dowolnej kolumny za pomocą operatora \texttt{\$}:

\begin{Shaded}
\begin{Highlighting}[]
\NormalTok{example\_data}\SpecialCharTok{$}\NormalTok{x}
\end{Highlighting}
\end{Shaded}

\begin{verbatim}
##  [1] 1 3 5 7 9 1 3 5 7 9
\end{verbatim}

\begin{Shaded}
\begin{Highlighting}[]
\FunctionTok{all.equal}\NormalTok{(}\FunctionTok{length}\NormalTok{(example\_data}\SpecialCharTok{$}\NormalTok{x),}
          \FunctionTok{length}\NormalTok{(example\_data}\SpecialCharTok{$}\NormalTok{y),}
          \FunctionTok{length}\NormalTok{(example\_data}\SpecialCharTok{$}\NormalTok{z))}
\end{Highlighting}
\end{Shaded}

\begin{verbatim}
## [1] TRUE
\end{verbatim}

\begin{Shaded}
\begin{Highlighting}[]
\FunctionTok{str}\NormalTok{(example\_data)}
\end{Highlighting}
\end{Shaded}

\begin{verbatim}
## 'data.frame':    10 obs. of  3 variables:
##  $ x: num  1 3 5 7 9 1 3 5 7 9
##  $ y: chr  "Hello" "Hello" "Hello" "Hello" ...
##  $ z: logi  TRUE FALSE TRUE FALSE TRUE FALSE ...
\end{verbatim}

\begin{Shaded}
\begin{Highlighting}[]
\FunctionTok{nrow}\NormalTok{(example\_data)}
\end{Highlighting}
\end{Shaded}

\begin{verbatim}
## [1] 10
\end{verbatim}

\begin{Shaded}
\begin{Highlighting}[]
\FunctionTok{ncol}\NormalTok{(example\_data)}
\end{Highlighting}
\end{Shaded}

\begin{verbatim}
## [1] 3
\end{verbatim}

\begin{Shaded}
\begin{Highlighting}[]
\FunctionTok{dim}\NormalTok{(example\_data)}
\end{Highlighting}
\end{Shaded}

\begin{verbatim}
## [1] 10  3
\end{verbatim}

\begin{Shaded}
\begin{Highlighting}[]
\FunctionTok{names}\NormalTok{(example\_data)}
\end{Highlighting}
\end{Shaded}

\begin{verbatim}
## [1] "x" "y" "z"
\end{verbatim}

\hypertarget{praca-z-data.frames}{%
\subsubsection{\texorpdfstring{Praca z
\texttt{data.frames}}{Praca z data.frames}}\label{praca-z-data.frames}}

Powyższa funkcja \texttt{data.frame()} to jeden ze sposobów tworzenia
ramki danych. Możemy także importować dane z różnych typów plików do
\texttt{R}, a także korzystać z danych przechowywanych w pakietach.

Aby wczytać te dane z powrotem do \texttt{R}, użyjemy wbudowanej funkcji
\texttt{read.csv}:

\begin{Shaded}
\begin{Highlighting}[]
\NormalTok{example\_data\_from\_disk }\OtherTok{=} \FunctionTok{read.csv}\NormalTok{(}\StringTok{"example{-}data.csv"}\NormalTok{)}
\NormalTok{example\_data\_from\_disk}
\end{Highlighting}
\end{Shaded}

\begin{verbatim}
##    x       y     z
## 1  1   Hello  TRUE
## 2  3   Hello FALSE
## 3  5   Hello  TRUE
## 4  7   Hello FALSE
## 5  9   Hello  TRUE
## 6  1   Hello FALSE
## 7  3   Hello  TRUE
## 8  5   Hello FALSE
## 9  7   Hello  TRUE
## 10 9 Goodbye FALSE
\end{verbatim}

Korzystając z danych, zazwyczaj chcielibyśmy zrobić trzy rzeczy:

\begin{itemize}
\tightlist
\item
  Spójrzeć na surowe dane.
\item
  Zrozumieć dane. (Skąd to się wzięło? Jakie są zmienne? Itp.)
\item
  Zwizualiować dane.
\end{itemize}

Aby przeglądać dane w \texttt{data.frame}, mamy dwa przydatne polecenia:
\texttt{head()} i \texttt{str()}.

\begin{Shaded}
\begin{Highlighting}[]
\CommentTok{\# pracujemy z wbudowanym zbiorem danych mtcars:}
\NormalTok{mtcars}
\end{Highlighting}
\end{Shaded}

\begin{verbatim}
##                      mpg cyl  disp  hp drat    wt  qsec vs am gear carb
## Mazda RX4           21.0   6 160.0 110 3.90 2.620 16.46  0  1    4    4
## Mazda RX4 Wag       21.0   6 160.0 110 3.90 2.875 17.02  0  1    4    4
## Datsun 710          22.8   4 108.0  93 3.85 2.320 18.61  1  1    4    1
## Hornet 4 Drive      21.4   6 258.0 110 3.08 3.215 19.44  1  0    3    1
## Hornet Sportabout   18.7   8 360.0 175 3.15 3.440 17.02  0  0    3    2
## Valiant             18.1   6 225.0 105 2.76 3.460 20.22  1  0    3    1
## Duster 360          14.3   8 360.0 245 3.21 3.570 15.84  0  0    3    4
## Merc 240D           24.4   4 146.7  62 3.69 3.190 20.00  1  0    4    2
## Merc 230            22.8   4 140.8  95 3.92 3.150 22.90  1  0    4    2
## Merc 280            19.2   6 167.6 123 3.92 3.440 18.30  1  0    4    4
## Merc 280C           17.8   6 167.6 123 3.92 3.440 18.90  1  0    4    4
## Merc 450SE          16.4   8 275.8 180 3.07 4.070 17.40  0  0    3    3
## Merc 450SL          17.3   8 275.8 180 3.07 3.730 17.60  0  0    3    3
## Merc 450SLC         15.2   8 275.8 180 3.07 3.780 18.00  0  0    3    3
## Cadillac Fleetwood  10.4   8 472.0 205 2.93 5.250 17.98  0  0    3    4
## Lincoln Continental 10.4   8 460.0 215 3.00 5.424 17.82  0  0    3    4
## Chrysler Imperial   14.7   8 440.0 230 3.23 5.345 17.42  0  0    3    4
## Fiat 128            32.4   4  78.7  66 4.08 2.200 19.47  1  1    4    1
## Honda Civic         30.4   4  75.7  52 4.93 1.615 18.52  1  1    4    2
## Toyota Corolla      33.9   4  71.1  65 4.22 1.835 19.90  1  1    4    1
## Toyota Corona       21.5   4 120.1  97 3.70 2.465 20.01  1  0    3    1
## Dodge Challenger    15.5   8 318.0 150 2.76 3.520 16.87  0  0    3    2
## AMC Javelin         15.2   8 304.0 150 3.15 3.435 17.30  0  0    3    2
## Camaro Z28          13.3   8 350.0 245 3.73 3.840 15.41  0  0    3    4
## Pontiac Firebird    19.2   8 400.0 175 3.08 3.845 17.05  0  0    3    2
## Fiat X1-9           27.3   4  79.0  66 4.08 1.935 18.90  1  1    4    1
## Porsche 914-2       26.0   4 120.3  91 4.43 2.140 16.70  0  1    5    2
## Lotus Europa        30.4   4  95.1 113 3.77 1.513 16.90  1  1    5    2
## Ford Pantera L      15.8   8 351.0 264 4.22 3.170 14.50  0  1    5    4
## Ferrari Dino        19.7   6 145.0 175 3.62 2.770 15.50  0  1    5    6
## Maserati Bora       15.0   8 301.0 335 3.54 3.570 14.60  0  1    5    8
## Volvo 142E          21.4   4 121.0 109 4.11 2.780 18.60  1  1    4    2
\end{verbatim}

Możesz zobaczyć, że to drukuje całą ramkę danych na ekranie. Funkcja
\texttt{head()} wyświetli pierwsze \texttt{n} obserwacji ramki danych.

\begin{Shaded}
\begin{Highlighting}[]
\FunctionTok{head}\NormalTok{(mtcars,}\AttributeTok{n=}\DecValTok{2}\NormalTok{)}
\end{Highlighting}
\end{Shaded}

\begin{verbatim}
##               mpg cyl disp  hp drat    wt  qsec vs am gear carb
## Mazda RX4      21   6  160 110  3.9 2.620 16.46  0  1    4    4
## Mazda RX4 Wag  21   6  160 110  3.9 2.875 17.02  0  1    4    4
\end{verbatim}

\begin{Shaded}
\begin{Highlighting}[]
\FunctionTok{head}\NormalTok{(mtcars) }\CommentTok{\# default}
\end{Highlighting}
\end{Shaded}

\begin{verbatim}
##                    mpg cyl disp  hp drat    wt  qsec vs am gear carb
## Mazda RX4         21.0   6  160 110 3.90 2.620 16.46  0  1    4    4
## Mazda RX4 Wag     21.0   6  160 110 3.90 2.875 17.02  0  1    4    4
## Datsun 710        22.8   4  108  93 3.85 2.320 18.61  1  1    4    1
## Hornet 4 Drive    21.4   6  258 110 3.08 3.215 19.44  1  0    3    1
## Hornet Sportabout 18.7   8  360 175 3.15 3.440 17.02  0  0    3    2
## Valiant           18.1   6  225 105 2.76 3.460 20.22  1  0    3    1
\end{verbatim}

Funkcja \texttt{str()} wyświetli „strukturę'' ramki danych. Wyświetli
liczbę \textbf{obserwacji} i \textbf{zmiennych}, wylistuje zmienne, poda
typ każdej zmiennej i pokaże niektóre elementy każdej zmiennej.
Informacje te można znaleźć także w oknie „Środowisko'' w RStudio.

\begin{Shaded}
\begin{Highlighting}[]
\FunctionTok{str}\NormalTok{(mtcars)}
\end{Highlighting}
\end{Shaded}

\begin{verbatim}
## 'data.frame':    32 obs. of  11 variables:
##  $ mpg : num  21 21 22.8 21.4 18.7 18.1 14.3 24.4 22.8 19.2 ...
##  $ cyl : num  6 6 4 6 8 6 8 4 4 6 ...
##  $ disp: num  160 160 108 258 360 ...
##  $ hp  : num  110 110 93 110 175 105 245 62 95 123 ...
##  $ drat: num  3.9 3.9 3.85 3.08 3.15 2.76 3.21 3.69 3.92 3.92 ...
##  $ wt  : num  2.62 2.88 2.32 3.21 3.44 ...
##  $ qsec: num  16.5 17 18.6 19.4 17 ...
##  $ vs  : num  0 0 1 1 0 1 0 1 1 1 ...
##  $ am  : num  1 1 1 0 0 0 0 0 0 0 ...
##  $ gear: num  4 4 4 3 3 3 3 4 4 4 ...
##  $ carb: num  4 4 1 1 2 1 4 2 2 4 ...
\end{verbatim}

W tym zbiorze danych obserwacja dotyczy konkretnego modelu samochodu, a
zmienne opisują cechy samochodu, na przykład jego zużycie paliwa lub
masę.

Aby lepiej zrozumieć zbiór danych, używamy operatora „?'', aby
wyświetlić dokumentację danych.

\begin{Shaded}
\begin{Highlighting}[]
\NormalTok{?mtcars}
\end{Highlighting}
\end{Shaded}

\texttt{R} posiada szereg funkcji umożliwiających szybką pracę i
wydobywanie podstawowych informacji z ramek danych. Aby szybko uzyskać
wektor nazw zmiennych, używamy funkcji \texttt{names()}.

\begin{Shaded}
\begin{Highlighting}[]
\FunctionTok{names}\NormalTok{(mtcars)}
\end{Highlighting}
\end{Shaded}

\begin{verbatim}
##  [1] "mpg"  "cyl"  "disp" "hp"   "drat" "wt"   "qsec" "vs"   "am"   "gear"
## [11] "carb"
\end{verbatim}

Aby uzyskać dostęp do jednej ze zmiennych \textbf{jako wektora}, używamy
operatora \texttt{\$}.

\begin{Shaded}
\begin{Highlighting}[]
\NormalTok{mtcars}\SpecialCharTok{$}\NormalTok{mpg}
\end{Highlighting}
\end{Shaded}

\begin{verbatim}
##  [1] 21.0 21.0 22.8 21.4 18.7 18.1 14.3 24.4 22.8 19.2 17.8 16.4 17.3 15.2 10.4
## [16] 10.4 14.7 32.4 30.4 33.9 21.5 15.5 15.2 13.3 19.2 27.3 26.0 30.4 15.8 19.7
## [31] 15.0 21.4
\end{verbatim}

\begin{Shaded}
\begin{Highlighting}[]
\NormalTok{mtcars}\SpecialCharTok{$}\NormalTok{wt}
\end{Highlighting}
\end{Shaded}

\begin{verbatim}
##  [1] 2.620 2.875 2.320 3.215 3.440 3.460 3.570 3.190 3.150 3.440 3.440 4.070
## [13] 3.730 3.780 5.250 5.424 5.345 2.200 1.615 1.835 2.465 3.520 3.435 3.840
## [25] 3.845 1.935 2.140 1.513 3.170 2.770 3.570 2.780
\end{verbatim}

Do uzyskania informacji o wymiarach ramki danych możemy użyć funkcji
\texttt{dim()}, \texttt{nrow()} i \texttt{ncol()}.

\begin{Shaded}
\begin{Highlighting}[]
\FunctionTok{dim}\NormalTok{(mtcars)}
\end{Highlighting}
\end{Shaded}

\begin{verbatim}
## [1] 32 11
\end{verbatim}

\begin{Shaded}
\begin{Highlighting}[]
\FunctionTok{nrow}\NormalTok{(mtcars)}
\end{Highlighting}
\end{Shaded}

\begin{verbatim}
## [1] 32
\end{verbatim}

\begin{Shaded}
\begin{Highlighting}[]
\FunctionTok{ncol}\NormalTok{(mtcars)}
\end{Highlighting}
\end{Shaded}

\begin{verbatim}
## [1] 11
\end{verbatim}

Tutaj „nrow()'' to także liczba obserwacji, która w większości
przypadków jest \emph{wielkością próby}.

Wywoływanie ramek danych może działać podobnie jak wywoływanie macierzy
za pomocą nawiasów kwadratowych, \texttt{{[}\ ,\ {]}}. Tutaj znajdujemy
pojazdy o mpg powyżej 25 mil na galon i wyświetlamy tylko kolumny
„cyl'', „disp'' i „wt''.

\begin{Shaded}
\begin{Highlighting}[]
\CommentTok{\# mpg[rząd warunek, kolumna warunek]}
\NormalTok{mtcars[mtcars}\SpecialCharTok{$}\NormalTok{mpg }\SpecialCharTok{\textgreater{}} \DecValTok{20}\NormalTok{, }\FunctionTok{c}\NormalTok{(}\StringTok{"cyl"}\NormalTok{, }\StringTok{"disp"}\NormalTok{, }\StringTok{"wt"}\NormalTok{)]}
\end{Highlighting}
\end{Shaded}

\begin{verbatim}
##                cyl  disp    wt
## Mazda RX4        6 160.0 2.620
## Mazda RX4 Wag    6 160.0 2.875
## Datsun 710       4 108.0 2.320
## Hornet 4 Drive   6 258.0 3.215
## Merc 240D        4 146.7 3.190
## Merc 230         4 140.8 3.150
## Fiat 128         4  78.7 2.200
## Honda Civic      4  75.7 1.615
## Toyota Corolla   4  71.1 1.835
## Toyota Corona    4 120.1 2.465
## Fiat X1-9        4  79.0 1.935
## Porsche 914-2    4 120.3 2.140
## Lotus Europa     4  95.1 1.513
## Volvo 142E       4 121.0 2.780
\end{verbatim}

Tutaj „nrow()'' to także liczba obserwacji, która w większości
przypadków jest \emph{wielkością próby}.

Alternatywą byłoby użycie funkcji „subset()'', która ma znacznie
bardziej czytelną składnię.

\begin{Shaded}
\begin{Highlighting}[]
\FunctionTok{subset}\NormalTok{(mtcars, }\AttributeTok{subset =}\NormalTok{ mpg }\SpecialCharTok{\textgreater{}} \DecValTok{25}\NormalTok{, }\AttributeTok{select =} \FunctionTok{c}\NormalTok{(}\StringTok{"cyl"}\NormalTok{, }\StringTok{"disp"}\NormalTok{, }\StringTok{"wt"}\NormalTok{))}
\end{Highlighting}
\end{Shaded}

\hypertarget{zadanie-5}{%
\paragraph{Zadanie 5}\label{zadanie-5}}

1.Ile obserwacji jest w \texttt{mtcars}? 1. Ile zmiennych? 1. Jaka jest
średnia wartość \texttt{mpg}? 1. Jaka jest średnia wartość \texttt{mpg}
dla samochodów mających więcej niż 4 cylindry, czyli z
\texttt{cyl\textgreater{}4}?

\hypertarget{podstawy-programowania}{%
\subsection{Podstawy Programowania}\label{podstawy-programowania}}

W tej sekcji zilustrujemy kilka ogólnych pojęć związanych z
programowaniem.

\hypertarget{zmienne}{%
\subsubsection{Zmienne}\label{zmienne}}

Z terminem \emph{zmienna} spotkaliśmy się już kilkukrotnie, ale głównie
w kontekście kolumny ramki danych. W programowaniu zmienna oznacza
\emph{obiekt}. Można to powiedzieć inaczej, mówiąc, że zmienna jest
nazwą lub \emph{etykietą} czegoś:

\begin{Shaded}
\begin{Highlighting}[]
\NormalTok{x }\OtherTok{=} \DecValTok{1}
\NormalTok{y }\OtherTok{=} \StringTok{"roses"}
\NormalTok{z }\OtherTok{=} \ControlFlowTok{function}\NormalTok{(x)\{}\FunctionTok{sqrt}\NormalTok{(x)\}}
\end{Highlighting}
\end{Shaded}

Tutaj \texttt{x} odnosi się do wartości \texttt{1}, \texttt{y} zawiera
ciąg „róże'', a \texttt{z} to nazwa funkcji, która oblicza \(\sqrt{x}\).
Zauważ, że argument „x'' funkcji różni się od argumentu „x'', który
właśnie zdefiniowaliśmy. Jest \textbf{lokalny} dla funkcji:

\begin{Shaded}
\begin{Highlighting}[]
\NormalTok{x}
\end{Highlighting}
\end{Shaded}

\begin{verbatim}
## [1] 1
\end{verbatim}

\begin{Shaded}
\begin{Highlighting}[]
\FunctionTok{z}\NormalTok{(}\DecValTok{9}\NormalTok{)}
\end{Highlighting}
\end{Shaded}

\begin{verbatim}
## [1] 3
\end{verbatim}

\hypertarget{warunki}{%
\subsubsection{Warunki}\label{warunki}}

Warunkowanie pomaga dostosowywć kod do różnych okoliczności. Opierając
się na tym, że „warunek'' ma wartość „PRAWDA'', Twój program wykona
jedną rzecz, a nie inną. Jest to najbardziej znane jako instrukcja
„if/else''. W \texttt{R} składnia if/else wygląda następująco:

\begin{Shaded}
\begin{Highlighting}[]
\ControlFlowTok{if}\NormalTok{ (}\AttributeTok{condition =} \ConstantTok{TRUE}\NormalTok{) \{}
\NormalTok{  some R code}
\NormalTok{\} }\ControlFlowTok{else}\NormalTok{ \{}
\NormalTok{  some other R code}
\NormalTok{\}}
\end{Highlighting}
\end{Shaded}

Na przykład,

\begin{Shaded}
\begin{Highlighting}[]
\NormalTok{x }\OtherTok{=} \DecValTok{1}
\NormalTok{y }\OtherTok{=} \DecValTok{3}
\ControlFlowTok{if}\NormalTok{ (x }\SpecialCharTok{\textgreater{}}\NormalTok{ y) \{  }\CommentTok{\# test if x \textgreater{} y}
  \CommentTok{\# if TRUE}
\NormalTok{  z }\OtherTok{=}\NormalTok{ x }\SpecialCharTok{*}\NormalTok{ y}
  \FunctionTok{print}\NormalTok{(}\StringTok{"x większe y"}\NormalTok{)}
\NormalTok{\} }\ControlFlowTok{else}\NormalTok{ \{}
  \CommentTok{\# if FALSE}
\NormalTok{  z }\OtherTok{=}\NormalTok{ x }\SpecialCharTok{+} \DecValTok{5} \SpecialCharTok{*}\NormalTok{ y}
  \FunctionTok{print}\NormalTok{(}\StringTok{"x mniejsze równe y"}\NormalTok{)}
\NormalTok{\}}
\end{Highlighting}
\end{Shaded}

\begin{verbatim}
## [1] "x mniejsze równe y"
\end{verbatim}

\begin{Shaded}
\begin{Highlighting}[]
\NormalTok{z}
\end{Highlighting}
\end{Shaded}

\begin{verbatim}
## [1] 16
\end{verbatim}

\hypertarget{pux119tle}{%
\subsubsection{Pętle}\label{pux119tle}}

Pętle są bardzo ważnym narzedziem programistycznym. Jak sama nazwa
wskazuje, w \emph{pętli}, program \emph{wielokrotnie} wykonuje pętlę po
zestawie instrukcji, dopóki jakiś warunek nie każe mu się zatrzymać.
Bardzo potężna, a jednocześnie prosta konstrukcja polega na tym, że
program może \emph{policzyć, ile kroków} już wykonał - co może być ważne
w przypadku wielu algorytmów. Składnia pętli „for'' (są inne) jest
następująca

\begin{Shaded}
\begin{Highlighting}[]
\ControlFlowTok{for}\NormalTok{ (ix }\ControlFlowTok{in} \DecValTok{1}\SpecialCharTok{:}\DecValTok{10}\NormalTok{)\{   }\CommentTok{\# does not have to be 1:10!}
  \CommentTok{\# loop body: gets executed each time}
  \CommentTok{\# the value of ix changes with each iteration}
\NormalTok{\}}
\end{Highlighting}
\end{Shaded}

Rozważmy na przykład prostą pętlę „for'', która po prostu wypisze na
ekranie wartość \emph{iteratora} (w tym przypadku zwanego „i''):

\begin{Shaded}
\begin{Highlighting}[]
\ControlFlowTok{for}\NormalTok{ (i }\ControlFlowTok{in} \DecValTok{1}\SpecialCharTok{:}\DecValTok{5}\NormalTok{)\{}
  \FunctionTok{print}\NormalTok{(i)}
\NormalTok{\}}
\end{Highlighting}
\end{Shaded}

\begin{verbatim}
## [1] 1
## [1] 2
## [1] 3
## [1] 4
## [1] 5
\end{verbatim}

Zauważ, że zamiast \texttt{1:5} moglibyśmy mieć \emph{dowolny} rodzaj
iterowalnej kolekcji:

\begin{Shaded}
\begin{Highlighting}[]
\ControlFlowTok{for}\NormalTok{ (i }\ControlFlowTok{in} \FunctionTok{c}\NormalTok{(}\StringTok{"mango"}\NormalTok{,}\StringTok{"banany"}\NormalTok{,}\StringTok{"jabłka"}\NormalTok{))\{}
  \FunctionTok{print}\NormalTok{(}\FunctionTok{paste}\NormalTok{(}\StringTok{"Uwielbiam "}\NormalTok{,i))  }\CommentTok{\# the paste function pastes together strings}
\NormalTok{\}}
\end{Highlighting}
\end{Shaded}

\begin{verbatim}
## [1] "Uwielbiam  mango"
## [1] "Uwielbiam  banany"
## [1] "Uwielbiam  jabłka"
\end{verbatim}

Często widzimy także \emph{zagnieżdżone} pętle, co sugeruje ich nazwa:

\begin{Shaded}
\begin{Highlighting}[]
\ControlFlowTok{for}\NormalTok{ (i }\ControlFlowTok{in} \DecValTok{2}\SpecialCharTok{:}\DecValTok{3}\NormalTok{)\{}
  \CommentTok{\# pierwsze zagnieżdżenie: dla każdego i}
  \ControlFlowTok{for}\NormalTok{ (j }\ControlFlowTok{in} \FunctionTok{c}\NormalTok{(}\StringTok{"mango"}\NormalTok{,}\StringTok{"banany"}\NormalTok{,}\StringTok{"jabłka"}\NormalTok{))\{}
    \CommentTok{\# drugir zagnieżdżenie: dla każdego j}
    \FunctionTok{print}\NormalTok{(}\FunctionTok{paste}\NormalTok{(}\StringTok{"Poproszę"}\NormalTok{,i,j,}\StringTok{"?"}\NormalTok{))}
\NormalTok{  \}}
\NormalTok{\}}
\end{Highlighting}
\end{Shaded}

\begin{verbatim}
## [1] "Poproszę 2 mango ?"
## [1] "Poproszę 2 banany ?"
## [1] "Poproszę 2 jabłka ?"
## [1] "Poproszę 3 mango ?"
## [1] "Poproszę 3 banany ?"
## [1] "Poproszę 3 jabłka ?"
\end{verbatim}

Należy tu zauważyć, że obliczenia można wykonywać za pomocą iteratorów
\emph{będąc wewnątrz pętli}.

\hypertarget{funkcje}{%
\subsubsection{Funkcje}\label{funkcje}}

Do tej pory korzystaliśmy z funkcji, ale tak naprawdę nie omawialiśmy
niektórych ich szczegółów. Funkcja to zestaw instrukcji, które wykonuje
dla nas „R'', podobnie jak te zebrane w pliku skryptu. Dobrą rzeczą jest
to, że funkcje są znacznie bardziej elastyczne niż skrypty, ponieważ
mogą zależeć od \emph{argumentów wejściowych}, które zmieniają sposób
zachowania funkcji. Oto jak zdefiniować funkcję:

\begin{Shaded}
\begin{Highlighting}[]
\NormalTok{function\_name }\OtherTok{\textless{}{-}} \ControlFlowTok{function}\NormalTok{(arg1,}\AttributeTok{arg2=}\NormalTok{wartosc\_domyslna)\{}
  \CommentTok{\# caiło funkcji {-} function body}
  \CommentTok{\# zrób rzeczy z arg1 i arg2}
  \CommentTok{\# możesz mieć dowolną liczbę argumentów}
  \CommentTok{\# lista poleceń do \textasciigrave{}R\textasciigrave{} może być wprowadzona tutaj}
  \CommentTok{\# ostatnia linia jest zwrócona}
\NormalTok{\}}
\end{Highlighting}
\end{Shaded}

A oto trywialny przykład definicji funkcji:

\begin{Shaded}
\begin{Highlighting}[]
\NormalTok{witaj }\OtherTok{\textless{}{-}} \ControlFlowTok{function}\NormalTok{(}\AttributeTok{imie =} \StringTok{"Lord Vader"}\NormalTok{)\{}
  \FunctionTok{paste}\NormalTok{(}\StringTok{"Witaj,"}\NormalTok{,imie)}
  \CommentTok{\# mozemy również napisać:}
  \CommentTok{\# return(paste("Wiatj",imie))}
\NormalTok{\}}

\CommentTok{\# wywołujemy funkcję, pisząc jej nazwę z okrągłymi nawiasami.}
\FunctionTok{witaj}\NormalTok{()}
\end{Highlighting}
\end{Shaded}

\begin{verbatim}
## [1] "Witaj, Lord Vader"
\end{verbatim}

Widzisz, że jeśli nie określisz argumentu „twoje\_imię'', „R'' powraca
do podanej wartości domyślnej. Wypróbuj teraz pod własnym imieniem!

Samo wpisanie nazwy funkcji zwraca nam rzeczywistą definicję, co czasami
jest przydatne:

\begin{Shaded}
\begin{Highlighting}[]
\NormalTok{witaj}
\end{Highlighting}
\end{Shaded}

\begin{verbatim}
## function(imie = "Lord Vader"){
##   paste("Witaj,",imie)
##   # mozemy również napisać:
##   # return(paste("Wiatj",imie))
## }
\end{verbatim}

Warto wziąć pod uwagę, że zanim zdefiniowaliśmy powyżej funkcję
„witaj'', „R'' nie wiedział, co zrobić, jeśli wywołałeś funkcję
„hello()''. Funkcja nie istniała! W tym sensie \emph{nauczyliśmy „R''
nowej sztuczki}. Ta funkcja tworzenia nowych możliwości poza językiem
podstawowym jest jedną z najpotężniejszych cech języków programowania.
Ogólnie rzecz biorąc, dobrą praktyką jest dzielenie kodu na kilka
mniejszych funkcji zamiast jednego długiego pliku skryptu. Dzięki temu
Twój kod będzie bardziej czytelny i łatwiej będzie wyśledzić błędy.

\hypertarget{zadanie-6}{%
\paragraph{Zadanie 6}\label{zadanie-6}}

\begin{enumerate}
\def\labelenumi{\arabic{enumi}.}
\tightlist
\item
  Napisz pętlę for, która odlicza od 10 do 1 i wypisuje na ekranie
  wartość iteratora.
\item
  Zmodyfikuj tę pętlę, aby zapisać „i iteracje do wykonania'', gdzie
  „i'' jest iteratorem
\item
  Zmodyfikuj tę pętlę tak, aby każda iteracja trwała około jednej
  sekundy. Można to osiągnąć dodając polecenie \texttt{Sys.sleep(1)} pod
  linią wyświetlającą ``i iteracji do wykonania''.
\end{enumerate}

warnings()

\end{document}
